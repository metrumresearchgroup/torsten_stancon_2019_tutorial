% Created 2019-08-09 Fri 21:25
% Intended LaTeX compiler: pdflatex
\documentclass[presentation, allowframebreaks]{beamer}
\usepackage[utf8]{inputenc}
\usepackage[T1]{fontenc}
\usepackage{graphicx}
\usepackage{grffile}
\usepackage{longtable}
\usepackage{wrapfig}
\usepackage{rotating}
\usepackage[normalem]{ulem}
\usepackage{amsmath}
\usepackage{textcomp}
\usepackage{amssymb}
\usepackage{capt-of}
\usepackage{hyperref}
\usepackage[newfloat]{minted}
\usepackage{caption}
\usetheme{default}
\usepackage{tikz}
\usetikzlibrary{positioning}
% \definecolor{MRGGreen}{rgb}{0, 0.350, 0.200}
\definecolor{MRGGreen}{rgb}{0, 0, 0.6}  % make consistent with other colors
\graphicspath{{./figures/}}
\setbeamertemplate{section in toc}[sections numbered]
\usetheme{default}
\setcounter{secnumdepth}{2}
\author{Charles Margossian, Yi Zhang}
\date{StanCon 2019, Cambridge UK \\ August 2019}
\title{Population and ODE-based models \\ using Stan and Torsten}
\AtBeginSection[]{\begin{frame}<beamer>\frametitle{Outline}\tableofcontents[currentsection,hideallsubsections,subsubsectionstyle=hide]\end{frame}}
\hypersetup{
 pdfauthor={Charles Margossian, Yi Zhang},
 pdftitle={Population and ODE-based models \\ using Stan and Torsten},
 pdfkeywords={},
 pdfsubject={},
 pdfcreator={Emacs 25.3.1 (Org mode 9.1.3)}, 
 pdflang={English}}
\begin{document}

\maketitle
\begin{frame}{Outline}
\setcounter{tocdepth}{1}
\tableofcontents
\end{frame}


\section{Course information}
\label{sec:orge75ebe8}
\begin{frame}[label={sec:org9a97511}]{}
\begin{block}{Instructors}
\begin{itemize}
\item Charles Margossian
\begin{itemize}
\item Columbia University, Department of Statistics
\end{itemize}
\item Yi Zhang
\begin{itemize}
\item Metrum Research Group
\end{itemize}
\end{itemize}
\end{block}
\begin{block}{TA}
\begin{itemize}
\item Ben Bales
\begin{itemize}
\item Columbia University
\end{itemize}
\end{itemize}
\end{block}
\end{frame}
\begin{frame}[label={sec:org08f9411}]{Outline}
\begin{block}{Day 1}
\begin{itemize}
\item Introduction and modeling framework
\item Pharmacometrics models
\item Ordinary differential equation(ODE) based models
\item Numerical ODE integrators
\end{itemize}
\end{block}
\begin{block}{Day 2}
\begin{itemize}
\item Population models
\item Group/Population ODE integrators and MPI parallelisation
\item Group/Population solvers and MPI parallelisation
\end{itemize}
\end{block}
\end{frame}
\begin{frame}[label={sec:org642d47c}]{Logistics}
METWORX\texttrademark{}, cloud-based modeling \& simulation platform by Metrum Research Group.
\begin{center}
  \includegraphics[width=3cm]{metworx}
\end{center}

\begin{figure}
  \centering
  \begin{minipage}{0.3\textwidth}
    \centering
    \includegraphics[width=0.9\textwidth]{metworx_efficiencies}
  \end{minipage}
  \begin{minipage}{0.3\textwidth}
    \centering
    \includegraphics[width=0.9\textwidth]{metworx_desktop}
  \end{minipage}
  \begin{minipage}{0.3\textwidth}
    \centering
    \includegraphics[width=0.9\textwidth]{metworx_decisionmakingtools}
  \end{minipage}
\end{figure}
\end{frame}

\begin{frame}[label={sec:org968181e}]{Logistics}
Workshop package
\begin{itemize}
\item R scripts and Stan files to do the exercises
\item These slides
\item Outline of the course
\end{itemize}

We will be using:
\begin{itemize}
\item Torsten v0.87
\item RStan v2.19.2
\item ggplot, plyr, tidyr, dplyr
\end{itemize}
\end{frame}

\section{Introduction and modeling framework | \footnotesize{Charles Margossian}}
\label{sec:org683d0bc}

\begin{frame}[label={sec:org13169d7}]{Preliminary question}
\begin{itemize}
\item Why Bayesian in a field such as pharmacometrics?
\item Example - \emph{Bayesian aggregation of average data: an application in drug development} \cite{Weber:2018}
\end{itemize}
\end{frame}

\begin{frame}[label={sec:org3294b82}]{Modeling framework}
\begin{block}{Box's loop}
\begin{center}
\includegraphics[width=0.7\columnwidth]{./figures/box_loop.pdf}
\label{orgc22193f}
\end{center}
\end{block}
\end{frame}


\begin{frame}[label={sec:orgafceb09}]{Inference}
\begin{itemize}
\item find the set of parameters consistent with our model and our data
\item approximate this set with draws from the posterior distribution
\end{itemize}
\end{frame}
\begin{frame}[label={sec:org659f024}]{Sampling algorithm}
\begin{itemize}
\item Use dynamic HMC to sample \(\pi (\theta | y)\)
\item Requires users the specify \(\log \pi(\theta, y) = \log \pi(y | \theta) + \log \pi(\theta)\)
\end{itemize}
\end{frame}
\begin{frame}[label={sec:orge3c8b8b}]{The "criticism" step}
This step can be broken up in two parts:
\begin{enumerate}
\item did we sample from the correct distribution?
\item does our model capture the characteristics of the data we care about?
\end{enumerate}
\end{frame}
\begin{frame}[label={sec:orgf6adaf4}]{Diagnosing the inference algorithm}
\begin{itemize}
\item look at the trace and the density plots
\item look at \(\hat R\) and effective number of samples
\item have any warning messages been issued, i.e. divergent transitions ?
\end{itemize}
\end{frame}
\begin{frame}[label={sec:orgb39f7f5}]{Example: fitting a linear model}
Likelihood:
\begin{align*}
  Y \sim \mathrm{Normal}(x \beta, \sigma^2)
\end{align*}

Prior:
\begin{align*}
  \beta \sim & \mathrm{Normal}(2, 1) \\
  \sigma^2 \sim & \mathrm{Normal}(1, 1)
\end{align*}
\end{frame}

\begin{frame}
  \begin{center}
  \includegraphics[width=4.5in]{figures/linear_summary.png}
  \end{center}
\end{frame}

\begin{frame}
  \begin{center}
  \includegraphics[width=4.5in]{figures/linear_trace.png}
  \end{center}
\end{frame}

\begin{frame}
  \begin{center}
  \includegraphics[width=4.5in]{figures/linear_dens_plot.png}
  \end{center}
\end{frame}

\begin{frame}
  \begin{center}
  \includegraphics[width=4.5in]{figures/linear_pred.png}
  \end{center}
\end{frame}

\begin{frame}{So, how can we improve the model?}
Likelihood:
\begin{align*}
  Y \sim \mathrm{Normal}(x \beta, \sigma^2)
\end{align*}

Prior:
\begin{align*}
  \beta \sim & \mathrm{Normal}(2, 1) \\
  \sigma^2 \sim & \mathrm{Normal}(1, 1)
\end{align*}
\end{frame}

\section{Models in pharmacometrics | \footnotesize{Charles Margossian}}
\label{sec:org62126be}
\begin{frame}[label={sec:org9b9ea6d}]{What is the effect of a treatment on a patient?}
\begin{itemize}
\item \emph{pharmacokinetics (PK)}: how is the drug absorbed in the body?
\item \emph{pharmacodynamics (PD)}: once it is absorbed, what are its effects?
\end{itemize}
\end{frame}

\begin{frame}[label={sec:org0b519da}]{Example: Two compartment model}
\begin{center}
  \includegraphics[width=4in]{TwoCptNice.png}
\end{center}
\end{frame}

\begin{frame}[label={sec:org5260673}]{Two compartment model}
\begin{align*}
   y_\mathrm{gut}' &= -k_a y_\mathrm{gut} \\
   y_\mathrm{cent}' &= k_a y_\mathrm{gut} - \left(\frac{CL}{V_\mathrm{cent}} + \frac{Q}{V_\mathrm{cent}} \right) y_\mathrm{cent} +  \frac{Q}{V_\mathrm{peri}} y_\mathrm{peri} \\
   y_\mathrm{peri}' &= \frac{Q}{V_\mathrm{cent}} y_\mathrm{cent} - \frac{Q}{V_\mathrm{peri}} y_\mathrm{peri}
   \label{eq:2Cpt}
\end{align*}
\end{frame}

\begin{frame}[label={sec:org8d28610}]{Example 2: Bone mineral density model from \cite{Peterson:2012}}
\begin{center}
  \includegraphics[width = 3in]{RiggsBoneModel.jpg}
\end{center}
\end{frame}

\begin{frame}[label={sec:org752420c}]{Two compartment model}
Denote \(\theta = \{CL, Q, VC, VP, K_a \}\), the ODE coefficients.
Then
$$ y' = f(y, t, \theta) $$

Given an initial condition \(y_0 = y(t_0)\), solving the above ODE gives us
the \textcolor{MRGGreen}\{\textit{natural evolution}\} of the system at any given time point.
\end{frame}

\begin{frame}[label={sec:orgc747b2d}]{The event schedule}
An event can be:
\begin{itemize}
\item \textcolor{MRGGreen}{a state changer}: an (exterior) intervention that alters the state of the system; for example a bolus dosing or the beginning of an infusion.
\item \textcolor{MRGGreen}{an observation}: a measurement of a quantity of interest at a certain time.
\end{itemize}
\end{frame}

\begin{frame}[label={sec:org779b6be}]{Drug concentration in a patient's blood}
\begin{center}
  \includegraphics[width=3.5in]{multiple_doses.png}
\end{center}
\end{frame}

\begin{frame}[label={sec:orgd357832}]{The event schedule}
\begin{itemize}
\item There is no general theory for the event schedule :(
\item The modeling software NONMEM\textregistered proposes a convention for pharmacometrics, which we adopt in Torsten.
\end{itemize}
\end{frame}

\begin{frame}[label={sec:orgb6b72fe}]{Torsten functions}
Torsten functions offers additional built-in functions to simulate data from a compartment model.
\begin{center}
  \includegraphics[width=1.5in]{torstenLogo.png}
\end{center}

Each Torsten function requires users to specify:
\begin{itemize}
\item a system of ODEs and a method to solve it.
\item An event schedule.
\end{itemize}
\end{frame}
\begin{frame}[fragile,label={sec:orga4bc910}]{Torsten functions}
 \begin{minted}[breaklines=true,fontsize=\footnotesize,breakanywhere=true]{stan}
matrix = pmx_solve_onecpt(real[] time, real[] amt, real[] rate,
                               real[] ii, int[] evid, int[] cmt,
                               real[] addl, int[] ss, real[] theta,
                               real[] biovar, real[] tlag);

matrix = pmx_solve_twocpt(real[] time, real[] amt, real[] rate,
                               real[] ii, int[] evid, int[] cmt,
                               real[] addl, int[] ss, real[] theta,
                               real[] biovar, real[] tlag);
\end{minted}
\begin{itemize}
\item Analytically solutions for the one/two cpt models.
\item Event schedule
\item ODE coefficients, e.g. \(\theta = \{CL, Q, VC, VP, ka \}\) for two-cpt model.
\item bio-availibility fraction and lag times.
\end{itemize}
\end{frame}

\begin{frame}[label={sec:orgbdd1f6a}]{Example}
\begin{block}{Clinical trial}
\begin{itemize}
\item Single patient
\item Bolus doses with 1200 mg, administered every 12 hours, for a total of 15 doses.
\item Many observations for the first, second, and last doses.
\item Additional observation every 12 hours.
\end{itemize}
\begin{block}{Note: the observation are plasma drug concentration measurement.}
\end{block}
\begin{block}{See \texttt{data/twoCpt.data.r}.}
\end{block}
\end{block}
\end{frame}

\begin{frame}[label={sec:orga14030d}]{Example}
\begin{block}{Model}
\begin{itemize}
\item two compartment model with first-order absorption
\item prior information based on clinical trial conducted on a large population
\item normal error for the plasma drug concentration measurement.
\end{itemize}
\end{block}
\end{frame}
\begin{frame}[fragile,label={sec:org3856eff}]{Example}
 \begin{block}{Prior}
\begin{minted}[breaklines=true,fontsize=\footnotesize,breakanywhere=true]{stan}
CL ~ lognormal(log(10), 0.25);
Q ~ lognormal(log(15), 0.5);
VC ~ lognormal(log(35), 0.25);
VP ~ lognormal(log(105), 0.5);
ka ~ lognormal(log(2.5), 1);
sigma ~ cauchy(0, 1);
\end{minted}
\end{block}

\begin{block}{Likelihood}
\begin{align*}
  \log(cObs) \sim \mathrm{Normal}\left( \log \left(\frac{y_2}{VC} \right), \sigma^2 \right)
\end{align*}
\emph{\textcolor{MRGGreen}{Exercise 1 \\ (a)} write and fit this model, using \texttt{twoCptModel.r} and  \texttt{model/twoCptModel.stan}.} \\
\emph{\textcolor{MRGGreen}{(b)} write a generated quantities block and do posterior predictive checks.}
\end{block}
\end{frame}

\begin{frame}[label={sec:orgc4206a6}]{Resources}
\begin{itemize}
\item Torsten repository: \url{https://github.com/metrumresearchgroup/Torsten}
\item Torsten User manual (on GitHub and in the workshop folder).
\end{itemize}
\end{frame}

\section{ODEs in Stan and Torsten | \footnotesize{Charles Margossian}}
\label{sec:orgc7c6be2}
\begin{frame}[label={sec:orged99f2d}]{Arsenal of tools}
\begin{center}
  \includegraphics[width=4.5in]{odeSolvers.png}
\end{center}

For some examples, see \cite{Margossian:2017}.
\begin{block}{}
\begin{itemize}
\item the "optimized - applicable" spectrum is a heuristic; counter-examples can be built.
\item coding effort may also be a criterion
\end{itemize}
\end{block}
\end{frame}
\begin{frame}[label={sec:orgaeb2c7f}]{Matrix exponential}
Consider a system of linear ODEs:
$$ y^\prime(t) = Ky(t) $$
where \(K\) is a constant matrix.

Then
$$ y(t) = e^{tK} y_0 $$
\end{frame}
\begin{frame}[label={sec:orgaa41c12}]{Matrix Exponential}
$$ e^{tK} = \sum_{n=0}^{\infty} \dfrac{(tK)^n}{n!} = I + tK + \frac{(tK)^2}{2} + \frac{(tK)^3}{3!} + ... $$
\end{frame}
\begin{frame}[label={sec:org5eb6aaa}]{Matrix Exponential}
For example, the two compartment model generates the following matrix:
\[ K = \begin{bmatrix}
       -ka & 0 & 0 \\
       ka & - (CL + Q) / Vc & Q / Vp \\
       0 & Q / V_c & - Q / V_p
     \end{bmatrix}
  \]
\end{frame}
\begin{frame}[fragile,label={sec:org65fcdd9}]{Linear ODE solver in Torsten}
 \begin{minted}[breaklines=true,fontsize=\footnotesize,breakanywhere=true]{stan}
matrix = pmx_solve_linode(real[] time, real[] amt, real[] rate,
                     real[] ii, int[] evid, int[] cmt,
                     real[] addl, int[] ss,
                     matrix K, real[] biovar, real[] tlag)
\end{minted}
\end{frame}


\begin{frame}[fragile,label={sec:org259582c}]{Numerical integrator}
 \begin{minted}[breaklines=true,fontsize=\footnotesize,breakanywhere=true]{stan}
real[ , ] pmx_integrate_ode_rk45(ODE_RHS, real[] y0, real t0, real[] ts, real[] theta, real[] x_r, int[] x_i, real rtol = 1.e-6, real atol = 1.e-6, int max_step = 1e6);
\end{minted}
\begin{itemize}
\item \texttt{ODE\_RHS}: ODE right-hand-side \(f\) in \(y' = f(y, t, \theta, x_r, x_i)\).
\item \texttt{y0}: initial condition at time \texttt{t0}.
\item \texttt{t0}: initial time.
\item \texttt{ts}: times at which we require a solution.
\item \texttt{theta}: parameters to be passed to \(f\).
\item \texttt{x\_r}: real data to be passed to \(f\).
\item \texttt{x\_i}: integer data to be passed to \(f\).
\item \texttt{rtol}, \texttt{atol}, and \texttt{max\_step} are optional control
parameters for \emph{relative tolerance}, \emph{absolute tolerance}, and \emph{max number of time steps}, respectively. Their default values have no theoretical justification.
\end{itemize}
\end{frame}
\begin{frame}[fragile,label={sec:org7b1c86b}]{System function}
 \begin{minted}[breaklines=true,fontsize=\footnotesize,breakanywhere=true]{stan}
functions {
  real[] system(real time, real[] y, 
                real[] theta, real[] x_r, int[] x_i) {
  real dydt[3];
  real CL = theta[1];
  real Q = theta[2];

  /* .... */

  return dydt;
  }
}
\end{minted}
\end{frame}
\begin{frame}[fragile,label={sec:org976a751}]{Torsten function}
 \begin{minted}[breaklines=true,fontsize=\footnotesize,breakanywhere=true]{stan}
matrix pmx_solve_rk45(ODE_system, int nCmt, real[] time, real[] amt, real[] rate, real[] ii, int[] evid, int[] cmt, real[] addl, int[] ss, real[] theta, real[] biovar, real[] tlag, real rel_tol, real abs_tol, int max_step);
\end{minted}
\\ \ \\

\emph{\textcolor{MRGGreen}{Exercise 2}: Write, fit, and diagnose the two compartment model using (a) the \texttt{pmx\_solve\_rk45} function
and (b) the \texttt{pmx\_solve\_linode} function. \\
Do the results agree? How does performance vary?}
\end{frame}

\section{Numerical ODE integrators | \footnotesize{Yi Zhang}}
\label{sec:org1eef964}
\label{org920b505}
\begin{frame}[label={sec:org49c2693}]{Nonlinear ODEs without analytical solution}
\begin{block}{kinetics of an autocatalytic reaction \cite{robertson_numerical_1966}}
The structure of the reactions is 
\begin{equation*}
A \xrightarrow{k_1} B,\quad
B+B \xrightarrow{k_2} C + B,\quad
B+C \xrightarrow{k_3} C + A,
\end{equation*}
where \(k_1\), \(k_2\), \(k_3\) are the rate
constants and \(A\), \(B\) and \(C\) are the chemical species
involved. The corresponding ODEs are
\begin{align*}
x_1' &= -k_1x_1 + k_3x_2x_3\\
x_2' &=  k_1x_1 - k_2x_2^2 - k_3x_2x_3\\
x_3' &=  k_2x_2^2
\end{align*}
Given \(k_1=0.04, k_2=3.0e7, k_3=1.0e4\), we make inference
regarding the initial condition for \(x_1(t=0)\).
\end{block}
\end{frame}
\begin{frame}[label={sec:orgffbcce3}]{Nonlinear ODEs without analytical solution}
\begin{align*}
x_1' &= -k_1x_1 + k_3x_2x_3\\
x_2' &=  k_1x_1 - k_2x_2^2 - k_3x_2x_3\\
x_3' &=  k_2x_2^2
\end{align*}
Given \(k_1=0.04, k_2=3.0e7, k_3=1.0e4\), we make inference
regarding the initial condition for \(x_1(t=0)\).
\begin{block}{Exercise 3}
Write Stan function for the above ODE's RHS.
\end{block}
\end{frame}

\begin{frame}[fragile,label={sec:orgc673f4f}]{Stan function for autocatalytic kinetics}
 \begin{align*}
x_1' &= -k_1x_1 + k_3x_2x_3\\
x_2' &=  k_1x_1 - k_2x_2^2 - k_3x_2x_3\\
x_3' &=  k_2x_2^2
\end{align*}

\begin{minted}[breaklines=true,fontsize=\footnotesize,breakanywhere=true]{stan}
functions{
  real[] reaction(real t, real[] x, real[] theta, real[] r, int[] i){
    real dxdt[3];
    real k1 = theta[1];
    real k2 = theta[2];
    real k3 = theta[3];
    dxdt[1] = -k1*x[1] + k3*x[2]*x[3];
    dxdt[2] =  k1*x[1] - k3*x[2]*x[3] - k2*(x[2])^2;
    dxdt[3] =  k2*(x[2])^2;
    return dxdt;
  }
}
\end{minted}
\begin{itemize}
\item What's the initial conditions for \(x_2\) and \(x_3\)?
\end{itemize}
\end{frame}

\begin{frame}[fragile,label={sec:orgbedfbcd}]{Numerical integrators}
 \begin{itemize}
\item Runge-Kutta 4th/5th (\texttt{rk45})
\begin{itemize}
\item non-stiff equations
\item Most popular, try this if you don't know the nature of the ODE, or what you're doing, or both.
\end{itemize}
\item Backward differentiation formula (\texttt{bdf})
\begin{itemize}
\item stiff equations
\item More expensive to use
\end{itemize}
\item Adams-Moulton
\begin{itemize}
\item non-stiff equations
\item higher-order of accuracy(do you really need it?)
\item scales better with number of steps
\end{itemize}
\end{itemize}
\end{frame}

\begin{frame}[fragile,label={sec:orgacb1aef}]{Numerical integrators}
 \begin{center}
\begin{tabular}{lll}
Integrators & Stan & Torsten\\
\hline
\texttt{rk45} & \texttt{integrate\_ode\_rk45} & \texttt{pmx\_integrate\_ode\_rk45}\\
\texttt{BDF} & \texttt{integrate\_ode\_bdf} & \texttt{pmx\_integrate\_ode\_bdf}\\
\texttt{Adams} & \texttt{integrate\_ode\_adams} & \texttt{pmx\_integrate\_ode\_adams}\\
\end{tabular}

\end{center}

\begin{minted}[breaklines=true,fontsize=\footnotesize,breakanywhere=true]{stan}
real[ , ] pmx_integrate_ode_rk45(ODE_RHS, real[] y0, real t0, real[] ts, real[] theta, real[] x_r, int[] x_i, real rtol = 1.e-6, real atol = 1.e-6, int max_step = 1e6);
\end{minted}
\begin{itemize}
\item \texttt{ODE\_RHS}: ODE right-hand-side \(f\) in \(y' = f(y, t, \theta, x_r, x_i)\).
\item \texttt{y0}: initial condition at time \texttt{t0}.
\item \texttt{t0}: initial time.
\item \texttt{ts}: times at which we require a solution.
\item \texttt{theta}: parameters to be passed to \(f\).
\item \texttt{x\_r}: real data to be passed to \(f\).
\item \texttt{x\_i}: integer data to be passed to \(f\).
\end{itemize}
\end{frame}


\begin{frame}[fragile,label={sec:org9bfa959}]{Model}
 \begin{itemize}
\item In each of 8 experiments performed \(x3\) is observed.
\item Hierarchical model for \(x0[1]\)
\end{itemize}
\begin{minted}[breaklines=true,fontsize=\footnotesize,breakanywhere=true]{stan}
model {
  y0_mu ~ lognormal(log(2.0), 0.5);
  for (i in 1:nsub) {
    y0_1[i] ~ lognormal(y0_mu, 0.5);    
  }
  sigma ~ cauchy(0, 0.5); 
  obs ~ lognormal(log(x3), sigma);
}
\end{minted}
\end{frame}

\begin{frame}[fragile,label={sec:orgf639330}]{Data}
 \begin{block}{Data available for the inference}
\begin{minted}[breaklines=true,fontsize=\footnotesize,breakanywhere=true]{stan}
data {
  int<lower=1> nsub;       /* nb. of subjects */
  int<lower=1> len[nsub];  /* nb. of results-extraction time points for each subject */
  int<lower=1> ntot;       /* total nb. of results-extraction time points */
  real ts[ntot];           /* concatenated array for results-extraction time points */
  real obs[ntot];          /* concatenated array for observed x3 */
}
\end{minted}
\end{block}
\end{frame}

\begin{frame}[fragile,label={sec:orgc613ea2}]{Exercise 4}
 \begin{block}{Given above data and model, write the rest of Stan code.}
\begin{itemize}
\item Hint: use \texttt{chem.stan} as template, also see \texttt{chem.data.R} and \texttt{chem.init.R}.
\item Reaction begins with \(A\)(on which is also what we'd
like to make inference), the other two spiecies are
non-existent at the beginning of the reaction.
\item Which numerical integrator are you using? Why?
\end{itemize}
\end{block}
\end{frame}

\begin{frame}[fragile,label={sec:org94f9094}]{Exercise 4}
 How to build \& run?
\begin{block}{Edit/Add \texttt{cmdstan/make/local}}
\begin{minted}[breaklines=true,fontsize=\footnotesize,breakanywhere=true]{sh}
TORSTEN_MPI = 1  # flag on torsten's MPI solvers
CXXFLAGS += -isystem /usr/local/include    # path to MPI library's headers
\end{minted}
\end{block}
\begin{block}{Build in \texttt{cmdstan}}
\begin{minted}[breaklines=true,fontsize=\footnotesize,breakanywhere=true]{sh}
make path_to_workshop/RScript/model/chemical_reactions/chem
\end{minted}
\end{block}
\begin{block}{Run}
\begin{minted}[breaklines=true,fontsize=\footnotesize,breakanywhere=true]{sh}
./chem sample adapt delta=0.95 random seed=1104508041 data file=chem.data.R init=chem.init.R
\end{minted}
\end{block}
\end{frame}

\section{Population models | \footnotesize{Charles Margossian}}
\label{sec:org99fe7e6}
\begin{frame}[label={sec:org5fd7dca}]{Data pooled into groups}
\begin{itemize}
\item sport measurements are grouped by players
\item people's voting intention can be grouped by states, social status, etc.
\item medical measurements can be grouped by patients, age groups, treatments, etc.
\end{itemize}
\end{frame}

\begin{frame}[fragile,label={sec:org711314f}]{Data pooled into groups}
 \begin{itemize}
\item medical measurements are grouped by patients
\begin{itemize}
\item Simulated with \texttt{mrgsolve} \url{https://mrgsolve.github.io/}
\end{itemize}
\end{itemize}
\begin{figure}
  \includegraphics[width = 7.5cm]{Dosing_regimes.png}
\end{figure}
\end{frame}

\begin{frame}[label={sec:org2266f23}]{Hierarchical model}
With a hierarchical model, we can
\begin{itemize}
\item do partial pooling.
\item estimate how similar the groups are to one another.
\item estimate individual parameters.
$$\theta = (\theta_1, ..., \theta_L) \sim p(\theta | \theta_\mathrm{pop}) $$
$$y = (y_1, ..., y_N) \sim p(y | \theta, x) $$
\end{itemize}
\end{frame}
\begin{frame}[label={sec:org2b94e72}]{Hierarchical model}
$$\theta = (\theta_1, ..., \theta_L) \sim p(\theta | \theta_\mathrm{pop}) $$
$$y = (y_1, ..., y_N) \sim p(y | \theta, x) $$
\begin{center}
\includegraphics[width=0.6\textwidth]{./figures/hierachical_model.pdf}
\end{center}
\end{frame}

\begin{frame}[label={sec:org79f8f64}]{Example 3: Hierarchical two compartment model}
Likelihood function:
\begin{align*}
  &\log \theta \sim \mathrm{Normal}(\log \theta_\mathrm{pop}, \Omega) \\ \\
  &\Omega = \left(\begin{array}{ccccc} 
                          \omega_1 & 0 & 0 & 0 & 0 \\
                          0 & \omega_2 & 0 & 0 & 0 \\
                          0 & 0 & \omega_3 & 0 & 0 \\
                          0 & 0 & 0 & \omega_4 & 0 \\
                          0 & 0 & 0 & 0 & \omega_5
                          \end{array} \right) \\ \\ \\
  &\log (cObs) \sim \mathrm{Normal}\left(\log \left(\frac{y_2}{VC} \right), \sigma^2 \right)
\end{align*}
\end{frame}

\begin{frame}[label={sec:org0bce690}]{Exercise 5: Write, fit, and diagnose a hierarchical two}
compartment model for a population of 10 patients.
Use \texttt{data/twoCptPop.data.r} and \texttt{twoCptPop.r}.\}
\begin{itemize}
\item \emph{Start by running 3 chains with 30 iterations.}
\item \emph{Do you get any warning messages?}
\end{itemize}
\end{frame}
\begin{frame}[fragile,label={sec:orgb1ff125}]{Divergent transitions}
 \begin{itemize}
\item \emph{Do you get any warning messages?}
\begin{minted}[breaklines=true,fontsize=\footnotesize,breakanywhere=true]{bash}
There were 29 divergent transitions after warmup.
\end{minted}
\item A divergent transition occurs when we fail to accurately compute a Hamiltonian trajectory.
\item This is because we \textit{approximate} trajectories.
\item Our sampler may not be refined enough to explore the entire typical set.
\end{itemize}
\end{frame}

\begin{frame}[label={sec:orgeb1419b}]{Divergent transitions}
Consider the following hierarchical model:

\begin{align*}
  \alpha_i \sim& \mathrm{Normal}(\mu, \sigma)  \\
  y_i \sim& p(y | \alpha_i)
\end{align*}
\end{frame}

\begin{frame}[label={sec:org36bbbb1}]{Divergent transitons}
$$ \alpha_i \sim \mathrm{Normal}(\mu, \sigma) $$

Fitting this model yields the following pairs plot:
\begin{center}
  \includegraphics[width=3in]{pairs_baseball2.png}
\end{center}
\end{frame}

\begin{frame}[label={sec:orgf9973ed}]{Divergent transitons}
\begin{center}
  \includegraphics[width=3in]{pairs_baseball2.png}
\end{center}

\begin{itemize}
\item This geometric shape is known as Neil's funnel \cite{Neil:2003}.
\item Its interactions with HMC is described in \cite{Betancourt:2015}.
\item It occurs in hierarchical models when we have sparse data and
a centered prior.
\end{itemize}
\end{frame}
\begin{frame}[label={sec:orgdcb5cbf}]{Reparameterization}
\begin{block}{Proposition}
Reparameterize the model to avoid the funnel shape.
We will do so by standardizing \(\alpha\).

$$ \alpha_{\mathrm{std}, i} := \frac{\alpha_i - \mu}{\sigma} $$

Then

$$ \alpha_\mathrm{std} \sim \mathrm{Normal}(0, 1)  $$
\end{block}
\end{frame}
\begin{frame}[label={sec:org21e4d68}]{Reparameterization}
Then
  $$ \alpha_i = \mu + \sigma \alpha_{\mathrm{std}, i} $$

Hence
  $$ y_i  \sim p(\mu + \sigma \alpha_{\mathrm{std}, i}) $$
\begin{itemize}
\item Same data generating process; but how does this affect the geometry of the posterior?
\end{itemize}
\end{frame}
\begin{frame}[label={sec:org6a34d08}]{Reparameterization}
Our model is a little more complicated than the above example:
\begin{itemize}
\item a lot of parameters (100 +)!
\item multiple population parameters and hierarchical structures.
\item these parameters follow a log normal distribution (so we need a pairs plot with \(\log \theta\)).
\end{itemize}
\end{frame}

\begin{frame}[label={sec:orga5806a1}]{Reparameterization}
\begin{center}
  \includegraphics[width=12cm]{twoCptPairs1}
\end{center}
\end{frame}

\begin{frame}[label={sec:orgb9d6a71}]{Reparameterization}
\emph{\textcolor{MRGGreen}{Exercise 6}}:
  Reparametrize the two compartment population model and fit it.
\begin{itemize}
\item First, work out the appropriate parametrization. You should start with \(\log \theta_i \sim \mathrm{Normal}(\theta_{\mathrm{pop}, i}, \omega)\)
\item Write, fit, and check the inference (run 100 chains).
\item What kind of predictive checks can we do?
\end{itemize}
\end{frame}

\begin{frame}[label={sec:org26379ca}]{Reparameterization}
Need:
\begin{itemize}
\item predictions at an individual level
\item predictions at a population level

As always, this comes down to properly writing the data generating process
in the generated quantities block.
\end{itemize}
\end{frame}
\begin{frame}[label={sec:orga0c93da}]{Individual predictions}
\begin{center}
  \includegraphics[width = 10cm]{PredictionPatient.png}
\end{center}
\end{frame}

\begin{frame}[label={sec:org012cb29}]{Population predictions}
\begin{center}
  \includegraphics[width = 10cm]{PredictionPopulation.png}
\end{center}
\end{frame}

\begin{frame}[label={sec:org7b77284}]{Further reading}
For a very good case study on hierarchical models, see,
Bob Carpenter's \emph{Pooling with Hierarchical Models for Repeated Binary Trials}

\url{https://mc-stan.org/users/documentation/case-studies/pool-binary-trials.html}
\end{frame}

\section{ODE group integrators | \footnotesize{Yi Zhang}}
\label{sec:org326293f}

\begin{frame}[fragile,label={sec:orgbc05972}]{ODE group integrators}
 \begin{center}
\begin{tabular}{ll}
Single ODE system & ODE group\\
\hline
\texttt{pmx\_integrate\_ode\_rk45} & \texttt{pmx\_integrate\_ode\_group\_rk45}\\
\texttt{pmx\_integrate\_ode\_bdf} & \texttt{pmx\_integrate\_ode\_group\_bdf}\\
\texttt{pmx\_integrate\_ode\_adams} & \texttt{pmx\_integrate\_ode\_group\_adams}\\
\end{tabular}

\end{center}

\begin{columns}
\begin{column}{0.45\columnwidth}
\begin{block}{Single ODE system}
\begin{minted}[breaklines=true,fontsize=\footnotesize,breakanywhere=true]{stan}
real[,]
pmx_integrate_ode_xxx(
      f,
      real[] y0, real t0,
      real[] ts,
      real[] theta,
      real[] x_r, int[] x_i,
      ...);
\end{minted}
\end{block}
\end{column}

\begin{column}{0.55\columnwidth}
\begin{block}{ODE group}
\begin{minted}[breaklines=true,fontsize=\footnotesize,breakanywhere=true]{stan}
matrix
pmx_integrate_ode_group_xxx(
     f,
     real[ , ] y0, real t0,
     int[] len, real[] ts,
     real[ , ] theta,
     real[ , ] x_r, int[ , ] x_i,
     ...);
\end{minted}
\end{block}
\end{column}
\end{columns}
\end{frame}

\begin{frame}[fragile,label={sec:org0107c40}]{ODE group integrators}
 \begin{columns}
\begin{column}{0.45\columnwidth}
\begin{block}{Single ODE system}
\begin{minted}[breaklines=true,fontsize=\footnotesize,breakanywhere=true]{stan}
real[ , ]
pmx_integrate_ode_xxx(
      f,
      real[] y0, real t0,
      real[] ts,
      real[] theta,
      real[] x_r, int[] x_i,
      ...);
\end{minted}
\end{block}
\end{column}

\begin{column}{0.55\columnwidth}
\begin{block}{ODE group}
\begin{minted}[breaklines=true,fontsize=\footnotesize,breakanywhere=true]{stan}
matrix
pmx_integrate_ode_group_xxx(
     f,
     real[ , ] y0, real t0,
     int[] len, real[] ts,
     real[ , ] theta,
     real[ , ] x_r, int[ , ] x_i,
     ...);
\end{minted}
\end{block}
\end{column}
\end{columns}
\begin{block}{}
\begin{itemize}
\item \texttt{len} specifies the length of data for each subject within
the above ragged arrays, and the size of \texttt{len} is the size
of the population.
\item The group integrators return a single matrix ragged
column-wise. The number of rows equals to the size of ODE system.
\end{itemize}
\end{block}
\end{frame}

\begin{frame}[label={sec:org71e1a23}]{Exercise 7}
\begin{block}{autocatalytic reaction model: ODE group version}
\begin{itemize}
\item Change the loop with the numerical integrator to use group
integrator.
\item Remeber the return of the group integrator is a matrix
\begin{itemize}
\item nb. of rows: nb. of states
\item nb. of cols: nb. of \emph{total} results-extraction time points.
\end{itemize}
\end{itemize}
\end{block}
\end{frame}

\begin{frame}[fragile,label={sec:org947ec52}]{Exercise 7}
 \begin{block}{Build and run}
\begin{itemize}
\item Edit/Add \texttt{cmdstan/make/local}
\end{itemize}
\begin{minted}[breaklines=true,fontsize=\footnotesize,breakanywhere=true]{sh}
TORSTEN_MPI = 1  # flag on torsten's MPI solvers
CXXFLAGS += -isystem /usr/local/include    # path to MPI library's headers
\end{minted}
\begin{itemize}
\item Build in \texttt{cmdstan}
\end{itemize}
\begin{minted}[breaklines=true,fontsize=\footnotesize,breakanywhere=true]{sh}
make ../example-models/chemical_reactions/chem_group
\end{minted}
\begin{itemize}
\item Run
\end{itemize}
\begin{minted}[breaklines=true,fontsize=\footnotesize,breakanywhere=true]{sh}
mpiexec -n 2 -l ./chem_group sample adapt delta=0.95 random seed=1104508041 data file=chem.data.R init=chem.init.R
\end{minted}
\end{block}
\end{frame}

\begin{frame}[fragile,label={sec:orgdc5fbe0}]{Exercise 7}
 \begin{itemize}
\item What does output say?
\item How many cores can you use until performance saturates? Why?
\item (optional)Can you do it using Stan's \texttt{map\_rect}? Is there a difference in style, output, and performance?
\end{itemize}
\end{frame}

\section{PMX population solvers | \footnotesize{Yi Zhang}}
\label{sec:org93466e8}

\begin{frame}[fragile,label={sec:org75ab7d5}]{PMX population solvers}
 \begin{center}
\begin{tabular}{ll}
Single ODE system & ODE group\\
\hline
\texttt{pmx\_solve\_rk45} & \texttt{pmx\_solve\_group\_rk45}\\
\texttt{pmx\_solve\_bdf} & \texttt{pmx\_solve\_group\_bdf}\\
\texttt{pmx\_solve\_adams} & \texttt{pmx\_solve\_group\_adams}\\
\end{tabular}

\end{center}

\begin{columns}
\begin{column}{0.45\columnwidth}
\begin{block}{Individual solvers}
\begin{minted}[breaklines=true,fontsize=\footnotesize,breakanywhere=true]{stan}
matrix
pmx_solve_bdf(f, int nCmt,
  real[] time, real[] amt,
  real[] rate, real[] ii,
  int[] evid, int[] cmt,
  real[] addl, int[] ss,
  real[] theta, real[] biovar,
  real[] tlag, real rel_tol,
  real abs_tol, int max_step);
\end{minted}
\end{block}
\end{column}

\begin{column}{0.55\columnwidth}
\begin{block}{Population solvers}
\begin{minted}[breaklines=true,fontsize=\footnotesize,breakanywhere=true]{stan}
matrix
pmx_solve_group_bdf(f, int nCmt,
  int[] len, real[] time,
  real[] amt, real[] rate,
  real[] ii, int[] evid,
  int[] cmt, real[] addl,
  int[] ss, real[ , ] theta,
  real[ , ] biovar, real[ , ] tlag,
  real rel_tol, real abs_tol,
  int max_step);
\end{minted}
\end{block}
\end{column}
\end{columns}
\end{frame}


\begin{frame}[fragile,label={sec:orgc6ad11b}]{PMX population solvers}
 \begin{minted}[breaklines=true,fontsize=\footnotesize,breakanywhere=true]{stan}
matrix
pmx_solve_group_bdf(f, int nCmt, int[] len, real[] time, real[] amt, real[] rate, real[] ii, int[] evid, int[] cmt, real[] addl, int[] ss, real[,] theta, real[,] biovar, real[,] tlag, real rel_tol, real abs_tol, int max_step);
\end{minted}

\begin{block}{}
\begin{figure}[htbp]
\centering
\includegraphics[width=0.6\textwidth]{./figures/group_solver_args.pdf}
\caption{arguments and output of \texttt{pmx\_solve\_group\_xxx}}
\end{figure}
\end{block}
\end{frame}

\begin{frame}[label={sec:orgb7e6aba}]{Time-to-event model}
We analyze the time to the first grade 2+ peripheral neuropathy
(PN) event in patients treated with an antibody-drug conjugate (ADC) delivering monomethyl auristatin E
(MMAE). We will simulate and analyze data using a simplified version of the
model reported in \cite{lu_time--event_2017}.
\begin{itemize}
\item Fauxlatuzumab vedotin 1.2 mg/kg IV boluses q3w \(\times\) 6 does.
\item 19 patients with 6 right-censored (simulated data).
\end{itemize}
\begin{columns}
\begin{column}{0.3\columnwidth}
\begin{block}{Model scheme}
\begin{center}
\includegraphics[width=0.9\columnwidth]{./figures/lu2017Model.pdf}
\end{center}
\end{block}
\end{column}
\begin{column}{0.7\columnwidth}
\begin{block}{Note}
\begin{itemize}
\item To keep things simpler, we use the simulated individual CL and V values, and only model PD part of the problem.
\item PN hazard is substantially delayed relative to PK exposure.
\item Hazard increases over time to an extent not completely described by P
\end{itemize}
\end{block}
\end{column}
\end{columns}
\end{frame}
\begin{frame}[label={sec:org37bed30}]{Likelihood}
Likelihood for time to first PN \(\ge\) 2 event in the \(i^{th}\) patient:
\begin{align*}
\lefteqn{L\left(\theta | t_{\text{PN},i}, \text{censor}_i, X_i\right)} \\
  &= \left\{ \begin{array}{ll}
     h_i\left(t_{\text{PN},i} | \theta, X_i\right) e^{-\int_0^{t_{\text{PN},i}} h_i\left(u | \theta, X_i\right) du}, &
    \text{censor}_i = 0 \\
     e^{-\int_0^{t_{\text{PN},i}} h_i\left(u | \theta, X_i\right) du}, &
     \text{censor}_i = 1
\end{array} \right.
\end{align*}
where
 \begin{align*}
   t_{\text{PN}} &\equiv \text{time to first PN $\ge$ 2 or right
     censoring event} \\
 \theta &\equiv \text{model parameters} \\
 X &\equiv \text{independent variables / covariates} \\
 \text{censor} &\equiv \left\{ \begin{array}{ll}
     1, & \text{PN $\ge$ 2 event is right censored} \\
     0, & \text{PN $\ge$ 2 event is observed} 
 \end{array} \right.
\end{align*}
One can see the expression
\begin{equation*}
  e^{-\int_0^{t_{\text{PN},i}} h_i\left(u | \theta, X_i\right) du}
\end{equation*}
as the survival function at time \(t\).
\end{frame}

\begin{frame}[label={sec:org8d4255a}]{ODEs}
Hazard of PN grade 2+ based on the Weibull distribution,
with drug effect proportional to effect site concentration of MMAE:
\begin{align*}
  h_j(t) &= \beta E_{\text{drug}j}(t)^\beta t^{(\beta - 1)} \\
  E_{\text{drug}j}(t) &= \alpha c_{ej}(t) \\
  c^\prime_{ej}(t) &= k_{e0} \left(c_j(t) - c_{ej}(t)\right).
\end{align*}

Overall ODE system including integration of the hazard function:
\begin{align}
  x_1^\prime &= -\frac{CL}{V} x_1 \\
  x_2^\prime &= k_{e0} \left(\frac{x_1}{V} - x_2\right) \\
  x_3^\prime &= h(t)
  \end{align}
where \(x_2(t) = c_e(t)\) and \(x_3(t) = \int_0^t h(u) du\) aka cumulative hazard.
\end{frame}

\begin{frame}[fragile,label={sec:org9c097ad}]{Exercise 8: write the ODE system}
 \begin{minted}[breaklines=true,fontsize=\footnotesize,breakanywhere=true]{stan}
functions{
    real[] oneCptPNODE(real t, real[] x, real[] parms, real[] x_r, int[] x_i){
    real dxdt[3];
    real CL = parms[1];
    real V = parms[2];
    real ke0 = parms[3];
    real alpha = parms[4];
    real beta = parms[5];
    real Edrug;
    real hazard;
    /* ... */
    return dxdt;
  }
}
\end{minted}
\end{frame}

\begin{frame}[fragile,label={sec:org8091cf7}]{Exercise 8: the ODE system}
 \begin{minted}[breaklines=true,fontsize=\footnotesize,breakanywhere=true]{stan}
real[] oneCptPNODE(real t, real[] x, real[] parms, real[] x_r, int[] x_i){
  real dxdt[3];
  real CL = parms[1];
  real V = parms[2];
  real ke0 = parms[3];
  real alpha = parms[4];
  real beta = parms[5];
  real Edrug;
  real hazard;

  dxdt[1] = -(CL / V) * x[1];
  dxdt[2] = ke0 * (x[1] / V - x[2]);
  Edrug = alpha * x[2];
  if(t == 0){
    hazard = 0;
  }else{
    hazard = beta * Edrug^beta * t^(beta - 1);
  }
  dxdt[3] = hazard;
  return dxdt;
}
\end{minted}
\end{frame}

\begin{frame}[fragile,label={sec:org3256d3e}]{Parameters}
 \begin{block}{parameters}
\begin{minted}[breaklines=true,fontsize=\footnotesize,breakanywhere=true]{stan}
parameters{
  real<lower = 0> ke0;
  real<lower = 0> alpha;
  real<lower = 0> beta;
}
transformed parameters{
  vector<lower = 0>[nPNObs] survObs;
  row_vector<lower = 0>[nPNObs] EdrugObs;
  vector<lower = 0>[nPNObs] hazardObs;
  vector<lower = 0>[nPNCens] survCens;
  matrix<lower = 0>[3, nt] x;
  real<lower = 0> parms[nId, 5];

  for(j in 1:nId) {
    parms[j, ] = {CL[j], V[j], ke0, alpha, beta};
  }
  /* ... */
}
\end{minted}
\end{block}
\end{frame}
\begin{frame}[fragile,label={sec:orgab9dd2b}]{Parameters}
 \begin{minted}[breaklines=true,fontsize=\footnotesize,breakanywhere=true]{stan}
transformed parameters{
  vector<lower = 0>[nPNObs] survObs;
  row_vector<lower = 0>[nPNObs] EdrugObs;
  vector<lower = 0>[nPNObs] hazardObs;
  vector<lower = 0>[nPNCens] survCens;
  matrix<lower = 0>[3, nt] x;
  real<lower = 0> parms[nId, 5];

  for(j in 1:nId) {
    parms[j, ] = {CL[j], V[j], ke0, alpha, beta};
  }
  /* ... */
}
\end{minted}
\begin{block}{Exercise 9}
\begin{itemize}
\item Use \texttt{pmx\_solve\_group\_rk45} to solve for \texttt{x}.
\item Write likelihood expressions for \texttt{survObs}, \texttt{EdrugObs}, \texttt{hazardObs}, and \texttt{survCens}.
\end{itemize}
\end{block}
\end{frame}

\begin{frame}[fragile,label={sec:org4c7a24e}]{Exercise 9}
 \begin{itemize}
\item Stan's \texttt{target} variable and user-defined likelihood.
\end{itemize}
\begin{minted}[breaklines=true,fontsize=\footnotesize,breakanywhere=true]{stan}
model{
  ke0 ~ normal(0, 0.0005);
  alpha ~ normal(0, 0.000003);
  beta ~ normal(0, 1.5);

  target += log(hazardObs .* survObs); // observed PN event log likelihood
  target += log(survCens); // censored PN event log likelihood
}
\end{minted}
\end{frame}

\begin{frame}[fragile,label={sec:org0b9039b}]{Exercise 9}
 \begin{block}{Edit/Add \texttt{cmdstan/make/local}}
\begin{minted}[breaklines=true,fontsize=\footnotesize,breakanywhere=true]{sh}
TORSTEN_MPI = 1         # flag on torsten's MPI solvers
CXXFLAGS += -isystem /usr/local/include # path to MPI library's headers
\end{minted}
\end{block}
\begin{block}{Build in \texttt{cmdstan}}
\begin{minted}[breaklines=true,fontsize=\footnotesize,breakanywhere=true]{sh}
make ../example-models/ttpn2/ttpn2_group
\end{minted}
\end{block}
\begin{block}{Run}
\begin{minted}[breaklines=true,fontsize=\footnotesize,breakanywhere=true]{sh}
mpiexec -n 4 -l ttpn2_group sample num_warmup=500 num_samples=500 data file=ttpn2.data2.R init=ttpn2.init.R
\end{minted}
\end{block}
\end{frame}

\begin{frame}[fragile,label={sec:orgf252be1}]{Exercise 9}
 \begin{itemize}
\item The parallel performance is not optimal, why?
\item Can you do it using Stan's \texttt{map\_rect}?
\end{itemize}
\end{frame}


%%%%%
\begin{frame}
  \begin{center}
    {\Large Concluding Remarks}
  \end{center}
\end{frame}

\begin{frame}
  \frametitle{Where does Stan fit in the pharmacometrician's toolkit?}

  Bayesian modeling can be implemented using an array of softwares:
  \begin{itemize}
    \item probabilistic programing languages: TensorFlow probability,
    PyMC3, Edward
    \item pharmacometrics softwares: NONMEM, Monolix, etc.
  \end{itemize}

\end{frame}

\begin{frame}
  \frametitle{Where does Stan fit in the pharmacometrician's toolkit?}
  There is synergy between the research we do, and the development
  of other softwares:
  \begin{itemize}
    \item PyMC3, Edward, and NONMEM implement Stan's No-U Turn Sampler.
    \item Torsten borrows many of NONMEM's conventions
  \end{itemize}
\end{frame}

\begin{frame}
  \frametitle{What do Stan and Torsten bring to the table?}
  
  Currently:
  \begin{itemize}
    \item a very flexible and expressive language
    \item algorithms that are efficient and fast for a full Bayesian inference, 
    and that warn you when they fail.
    \item Diagnostic tools
    \item It's free and open-source
  \end{itemize}

  Our goals:
  \begin{itemize}
    \item More expressive features: PDEs and SDEs
    \item Algorithms for fast approximate Bayesian inference
    (variational inference, nested Laplace).
    \item High performance tools: GPU, within solver parallelization.
  \end{itemize}
\end{frame}

\begin{frame}
  \frametitle{What we covered}
  
  \begin{itemize}
    \item Writing and fitting compartment models with Torsten
    \item Defining ODEs and picking a numerical solver
    \item Building and parameterizing a population model
    \item Within-chain parallelization for population models
  \end{itemize}
\end{frame}

\begin{frame}
  \frametitle{What we didn't cover}
  
  \begin{itemize}
    \item Computing steady states with an algebraic solver.
    \item Combining multiple solvers, e.g. analytical and numerical methods
    \item More elaborate problems that combine all the moving parts we went through
  \end{itemize}
\end{frame}

\begin{frame}
  \frametitle{Where can I learn more?}
     \begin{itemize}
     \item The Stan book and the Torsten manual
     \item Bill Gillespie's workshop: \textit{Advanced use of Stan, Rstan, and Torsten
     for pharmacometric applications}
     \item Contributions to the Stan Conference: \url{https://github.com/stan-dev/stancon talks}
     \item Online tutorials: \url{https://mc-stan.org/users/documentation/tutorials.html}
     \item Betancourt's case studies: \url{https://betanalpha.github.io/writing/}
   \end{itemize}
\end{frame}

\begin{frame}
  \frametitle{Acknowledgments}

   \ \\ Torsten development team:
  \begin{itemize}
    \item Bill Gillespie
  \end{itemize}
  
  \ \\ Stan development team:
  \begin{itemize}
    \item Sebastian Weber
     \item Andrew Gelman
    \item Michael Betancourt
    \item Bob Carpenter
  \end{itemize}
  
  \ \\ Institututions:
  \begin{itemize}
    \item Metrum Research Group
    \item Columbia University
  \end{itemize}

\end{frame}

\begin{frame}[allowframebreaks]{Reference}
\bibliographystyle{apalike}
\footnotesize \bibliography{./ref}
\end{frame}
\end{document}
