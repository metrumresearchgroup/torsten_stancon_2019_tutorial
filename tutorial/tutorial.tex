% Created 2019-08-09 Fri 15:48
% Intended LaTeX compiler: pdflatex
\documentclass[presentation, allowframebreaks]{beamer}
\usepackage[utf8]{inputenc}
\usepackage[T1]{fontenc}
\usepackage{graphicx}
\usepackage{grffile}
\usepackage{longtable}
\usepackage{wrapfig}
\usepackage{rotating}
\usepackage[normalem]{ulem}
\usepackage{amsmath}
\usepackage{textcomp}
\usepackage{amssymb}
\usepackage{capt-of}
\usepackage{hyperref}
\usepackage[newfloat]{minted}
\usepackage{caption}
\usetheme{default}
\usepackage{tikz}
\usetikzlibrary{positioning}
\definecolor{MRGGreen}{rgb}{0, 0.350, 0.200}
\graphicspath{{./figures/}}
\setbeamertemplate{section in toc}[sections numbered]
\usetheme{default}
\setcounter{secnumdepth}{2}
\author{Charles Margossian, Yi Zhang}
\date{StanCon 2019, Cambridge UK \\ August 2019}
\title{Population and ODE-based models \\ using Stan and Torsten}
\AtBeginSection[]{\begin{frame}<beamer>\frametitle{Outline}\tableofcontents[currentsection,hideallsubsections,subsubsectionstyle=hide]\end{frame}}
\hypersetup{
 pdfauthor={Charles Margossian, Yi Zhang},
 pdftitle={Population and ODE-based models \\ using Stan and Torsten},
 pdfkeywords={},
 pdfsubject={},
 pdfcreator={Emacs 25.3.1 (Org mode 9.1.3)}, 
 pdflang={English}}
\begin{document}

\maketitle
\begin{frame}{Outline}
\setcounter{tocdepth}{1}
\tableofcontents
\end{frame}


\section{Course information}
\label{sec:orgcb98c01}
\label{org2810f74}
\begin{frame}[label={sec:orgd9f054d}]{}
\begin{block}{Instructors}
\begin{itemize}
\item Charles Margossian
\begin{itemize}
\item Columbia University, Department of Statistics
\end{itemize}
\item Yi Zhang
\begin{itemize}
\item Metrum Research Group
\end{itemize}
\end{itemize}
\end{block}
\begin{block}{TA}
\begin{itemize}
\item Steve Bronder
\begin{itemize}
\item Capital One
\end{itemize}
\end{itemize}
\end{block}
\end{frame}
\begin{frame}[label={sec:orgf9ed4d6}]{Outline}
\begin{block}{Day 1}
\begin{itemize}
\item Introduction and modeling framework
\item Pharmacometrics models
\item Ordinary differential equation(ODE) based models
\item Numerical ODE integrators
\end{itemize}
\end{block}
\begin{block}{Day 2}
\begin{itemize}
\item Population models
\item Group/Population ODE integrators and MPI parallelisation
\item Group/Population solvers and MPI parallelisation
\end{itemize}
\end{block}
\end{frame}
\begin{frame}[label={sec:orga4afb06}]{Logistics}
METWORX\texttrademark{}, cloud-based modeling \& simulation platform by Metrum Research Group.
\begin{center}
  \includegraphics[width=3cm]{metworx}
\end{center}

\begin{figure}
  \centering
  \begin{minipage}{0.3\textwidth}
    \centering
    \includegraphics[width=0.9\textwidth]{metworx_efficiencies}
  \end{minipage}
  \begin{minipage}{0.3\textwidth}
    \centering
    \includegraphics[width=0.9\textwidth]{metworx_desktop}
  \end{minipage}
  \begin{minipage}{0.3\textwidth}
    \centering
    \includegraphics[width=0.9\textwidth]{metworx_decisionmakingtools}
  \end{minipage}
\end{figure}
\end{frame}

\begin{frame}[label={sec:org6092cb0}]{Logistics}
Workshop package
\begin{itemize}
\item R scripts and Stan files to do the exercises
\item These slides
\item Outline of the course
\item pAdditional documentation
\end{itemize}

We will be using:
\begin{itemize}
\item Torsten v0.87
\item RStan v2.19.2
\item ggplot, plyr, tidyr, dplyr
\end{itemize}
\end{frame}

\section{Introduction and modeling framework | \footnotesize{Charles Margossian}}
\label{sec:org4d896eb}

\begin{frame}[label={sec:org843a7d5}]{Preliminary question}
\begin{itemize}
\item Why Bayesian in a field such as pharmacometrics?
\item Example - \emph{Bayesian aggregation of average data: an application in drug development} \cite{Weber:2018}
\end{itemize}
\end{frame}

\begin{frame}[label={sec:org41c1d22}]{Modeling framework}
\begin{block}{Box's loop}
\begin{center}
\includegraphics[width=0.7\columnwidth]{./figures/box_loop.pdf}
\label{orga960693}
\end{center}
\end{block}
\end{frame}


\begin{frame}[label={sec:orge7dcbcd}]{Inference}
\begin{itemize}
\item find the set of parameters consistent with our model and our data
\item approximate this set with draws from the posterior distribution
\end{itemize}
\end{frame}
\begin{frame}[label={sec:org2c5a54b}]{Sampling algorithm}
\begin{itemize}
\item Use the NUTS to sample \(\pi (\theta | y)\)
\item Requires users the specify \(\log \pi(\theta, y) = \log \pi(y | \theta) + \log \pi(\theta)\)
\end{itemize}
\end{frame}
\begin{frame}[label={sec:org43ba753}]{The "criticism" step}
This step can be broken up in two parts:
\begin{enumerate}
\item did we sample from the correct distribution?
\item does our model capture the characteristics of the data we care about?
\end{enumerate}
\end{frame}
\begin{frame}[label={sec:org99f0bad}]{Diagnosing the inference algorithm}
\begin{itemize}
\item look at the trace and the density plots
\item look at \(\hat R\) and effective number of samples
\item have any warning messages been issued, i.e. divergent transitions ?
\end{itemize}
\end{frame}
\begin{frame}[label={sec:orgaaf9965}]{Example: fitting a linear model}
Likelihood:
\begin{align*}
  Y \sim \mathrm{Normal}(x \beta, \sigma^2)
\end{align*}

Prior:
\begin{align*}
  \beta \sim & \mathrm{Normal}(2, 1) \\
  \sigma^2 \sim & \mathrm{Normal}(1, 1)
\end{align*}
\end{frame}

\section{Models in pharmacometrics | \footnotesize{Charles Margossian}}
\label{sec:orgfb532af}
\begin{frame}[label={sec:org5327d8f}]{What is the effect of a treatment on a patient?}
\begin{itemize}
\item \emph{pharmacokinetics (PK)}: how is the drug absorbed in the body?
\item \emph{pharmacodynamics (PD)}: once it is absorbed, what are its effects?
\end{itemize}
\end{frame}

\begin{frame}[label={sec:org24fe739}]{Example: Two compartment model}
\begin{center}
  \includegraphics[width=4in]{TwoCptNice.png}
\end{center}
\end{frame}

\begin{frame}[label={sec:org414d1a1}]{Two compartment model}
\begin{align*}
   y_\mathrm{gut}' &= -k_a y_\mathrm{gut} \\
   y_\mathrm{cent}' &= k_a y_\mathrm{gut} - \left(\frac{CL}{V_\mathrm{cent}} + \frac{Q}{V_\mathrm{cent}} \right) y_\mathrm{cent} +  \frac{Q}{V_\mathrm{peri}} y_\mathrm{peri} \\
   y_\mathrm{peri}' &= \frac{Q}{V_\mathrm{cent}} y_\mathrm{cent} - \frac{Q}{V_\mathrm{peri}} y_\mathrm{peri}
   \label{eq:2Cpt}
\end{align*}
\end{frame}

\begin{frame}[label={sec:org41688a9}]{Example 2: Bone mineral density model from \cite{Peterson:2012}}
\begin{center}
  \includegraphics[width = 3in]{RiggsBoneModel.jpg}
\end{center}
\end{frame}

\begin{frame}[label={sec:orgc07fe71}]{Two compartment model}
Denote \(\theta = \{CL, Q, VC, VP, K_a \}\), the ODE coefficients.
Then
$$ y' = f(y, t, \theta) $$

Given an initial condition \(y_0 = y(t_0)\), solving the above ODE gives us
the \textcolor{MRGGreen}\{\textit{natural evolution}\} of the system at any given time point.
\end{frame}

\begin{frame}[label={sec:org7c9d69a}]{The event schedule}
An event can be:
\begin{itemize}
\item \textcolor{MRGGreen}{Sate changer}: an (exterior) intervention that alters the state of the system; for example a bolus dosing or the beginning of an infusion.
\item \textcolor{MRGGreen}{Observation}: a measurement of a quantity of interest at a certain time.
\end{itemize}
\end{frame}

\begin{frame}[label={sec:orgbd33165}]{Drug concentration in a patient's blood}
\begin{center}
  \includegraphics[width=3.5in]{multiple_doses.png}
\end{center}
\end{frame}

\begin{frame}[label={sec:orgcba1182}]{The event schedule}
\begin{itemize}
\item There is no general theory for the event schedule :(
\item The modeling software NONMEM\textregistered proposes a convention for pharmacometrics, which we adopt in Torsten.
\end{itemize}
\end{frame}

\begin{frame}[label={sec:org0a231d5}]{Torsten functions}
Torsten functions offers additional built-in functions to simulate data from a compartment model.
\begin{center}
  \includegraphics[width=1.5in]{torstenLogo.png}
\end{center}

Each Torsten function requires users to specify:
\begin{itemize}
\item a system of ODEs and a method to solve it.
\item An event schedule.
\end{itemize}
\end{frame}
\begin{frame}[fragile,label={sec:org9e107b1}]{Torsten functions}
 \begin{minted}[breaklines=true,fontsize=\footnotesize,breakanywhere=true]{stan}
matrix = pmx_solve_onecpt(real[] time, real[] amt, real[] rate,
                               real[] ii, int[] evid, int[] cmt,
                               real[] addl, int[] ss, real[] theta,
                               real[] biovar, real[] tlag);

matrix = pmx_solve_twocpt(real[] time, real[] amt, real[] rate,
                               real[] ii, int[] evid, int[] cmt,
                               real[] addl, int[] ss, real[] theta,
                               real[] biovar, real[] tlag);
\end{minted}
\begin{itemize}
\item Analytically solutions for the one/two cpt models.
\item Event schedule
\item ODE coefficients, e.g. \(\theta = \{CL, Q, VC, VP, ka \}\) for two-cpt model.
\item bio-availibility fraction and lag times.
\end{itemize}
\end{frame}

\begin{frame}[label={sec:orgf2e9dbb}]{Example}
\begin{block}{Clinical trial}
\begin{itemize}
\item Single patient
\item Bolus doses with 1200 mg, administered every 12 hours, for a total of 15 doses.
\item Many observations for the first, second, and last doses.
\item Additional observation every 12 hours.
\end{itemize}
\begin{block}{Note: the observation are plasma drug concentration measurement.}
\end{block}
\begin{block}{See \texttt{data/twoCpt.data.r}.}
\end{block}
\end{block}
\end{frame}

\begin{frame}[label={sec:org26fe193}]{Example}
\begin{block}{Model}
\begin{itemize}
\item two compartment model with first-order absorption
\item prior information based on clinical trial conducted on a large population
\item normal error for the plasma drug concentration measurement.
\end{itemize}
\end{block}
\end{frame}
\begin{frame}[fragile,label={sec:orgcd207a8}]{Example}
 \begin{block}{Prior}
\begin{minted}[breaklines=true,fontsize=\footnotesize,breakanywhere=true]{stan}
CL ~ lognormal(log(10), 0.25);
Q ~ lognormal(log(15), 0.5);
VC ~ lognormal(log(35), 0.25);
VP ~ lognormal(log(105), 0.5);
ka ~ lognormal(log(2.5), 1);
sigma ~ cauchy(0, 1);
\end{minted}
\end{block}

\begin{block}{Likelihood}
\begin{align*}
  \log(cObs) \sim \mathrm{Normal}\left( \log \left(\frac{y_2}{VC} \right), \sigma^2 \right)
\end{align*}
\emph{\textcolor{MRGGreen}{Exercise 1}: write and fit this model, using \texttt{twoCptModel.r} and  \texttt{model/twoCptModel.stan}.}
\emph{\textcolor{MRGGreen}{Exercise 2}: Write a generated quantities block and do posterior predictive checks.}
\end{block}
\end{frame}

\begin{frame}[label={sec:org2b167e4}]{Resources}
\begin{itemize}
\item Torsten repository: \url{https://github.com/metrumresearchgroup/Torsten}
\item Torsten User manual (on GitHub and in the workshop folder).
\end{itemize}
\end{frame}

\section{ODEs in Stan and Torsten | \footnotesize{Charles Margossian}}
\label{sec:org8e3f90b}
\begin{frame}[label={sec:org252bd32}]{Arsenal of tools}
\begin{center}
  \includegraphics[width=4.5in]{odeSolvers.png}
\end{center}

For some examples, see \cite{Margossian:2017}.
\begin{block}{}
\begin{itemize}
\item the "optimized - applicable" spectrum is a heuristic; counter-examples can be built.
\item coding effort may also be a criterion
\end{itemize}
\end{block}
\end{frame}
\begin{frame}[label={sec:org8a15147}]{Matrix exponential}
Consider a system of linear ODEs:
$$ y^\prime(t) = Ky(t) $$
where \(K\) is a constant matrix.

Then
$$ y(t) = e^{tK} y_0 $$
\end{frame}
\begin{frame}[label={sec:org44ff4ff}]{Matrix Exponential}
$$ e^{tK} = \sum_{n=0}^{\infty} \dfrac{(tK)^n}{n!} = I + tK + \frac{(tK)^2}{2} + \frac{(tK)^3}{3!} + ... $$
\end{frame}
\begin{frame}[label={sec:org0e1779b}]{Matrix Exponential}
For example, the two compartment model generates the following matrix:
\[ K = \begin{bmatrix}
       -ka & 0 & 0 \\
       ka & - (CL + Q) / Vc & Q / Vp \\
       0 & Q / V_c & - Q / V_p
     \end{bmatrix}
  \]
\end{frame}
\begin{frame}[fragile,label={sec:org9c7f904}]{Linear ODE solver in Torsten}
 \begin{minted}[breaklines=true,fontsize=\footnotesize,breakanywhere=true]{stan}
matrix = pmx_solve_linode(real[] time, real[] amt, real[] rate,
                     real[] ii, int[] evid, int[] cmt,
                     real[] addl, int[] ss,
                     matrix K, real[] biovar, real[] tlag)
\end{minted}
\end{frame}
\begin{frame}[fragile,label={sec:orgab3c852}]{Numerical integrator}
 \begin{minted}[breaklines=true,fontsize=\footnotesize,breakanywhere=true]{stan}
real[ , ] pmx_integrate_ode_rk45(ODE_RHS, real[] y0, real t0, real[] ts, real[] theta, real[] x_r, int[] x_i, real rtol = 1.e-6, real atol = 1.e-6, int max_step = 1e6);
\end{minted}
\begin{itemize}
\item \texttt{ODE\_RHS}: ODE right-hand-side \(f\) in \(y' = f(y, t, \theta, x_r, x_i)\).
\item \texttt{y0}: initial condition at time \texttt{t0}.
\item \texttt{t0}: initial time.
\item \texttt{ts}: times at which we require a solution.
\item \texttt{theta}: parameters to be passed to \(f\).
\item \texttt{x\_r}: real data to be passed to \(f\).
\item \texttt{x\_i}: integer data to be passed to \(f\).
\item \texttt{rtol}, \texttt{atol}, and \texttt{max\_step} are optional control
parameters for \emph{relative tolerance}, \emph{absolute tolerance}, and \emph{max number of time steps}, respectively. Their default values have no theoretical justification.
\end{itemize}
\end{frame}
\begin{frame}[fragile,label={sec:org5d4074d}]{System function}
 \begin{minted}[breaklines=true,fontsize=\footnotesize,breakanywhere=true]{stan}
functions {
  real[] system(real time, real[] y, 
                real[] theta, real[] x_r, int[] x_i) {
  real dydt[3];
  real CL = theta[1];
  real Q = theta[2];

  /* .... */

  return dydt;
  }
}
\end{minted}
\end{frame}
\begin{frame}[fragile,label={sec:orgcbca924}]{Torsten function}
 \begin{minted}[breaklines=true,fontsize=\footnotesize,breakanywhere=true]{stan}
matrix pmx_solve_rk45(ODE_system, int nCmt, real[] time, real[] amt, real[] rate, real[] ii, int[] evid, int[] cmt, real[] addl, int[] ss, real[] theta, real[] biovar, real[] tlag, real rel_tol, real abs_tol, int max_step);
\end{minted}
\emph{\textcolor{MRGGreen}{Exercise 3}: Write, fit, and diagnose the two compartment model using the \texttt{pmx\_solve\_rk45} function.}
\end{frame}

\section{Numerical ODE integrators | \footnotesize{Yi Zhang}}
\label{sec:org9e98cfa}

\begin{frame}[label={sec:org9fcf4d0}]{Nonlinear ODEs without analytical solution}
\begin{block}{kinetics of an autocatalytic reaction \cite{robertson_numerical_1966}}
The structure of the reactions is 
\begin{equation*}
A \xrightarrow{k_1} B,\quad
B+B \xrightarrow{k_2} C + B,\quad
B+C \xrightarrow{k_3} C + A,
\end{equation*}
where \(k_1\), \(k_2\), \(k_3\) are the rate
constants and \(A\), \(B\) and \(C\) are the chemical species
involved. The corresponding ODEs are
\begin{align*}
x_1' &= -k_1x_1 + k_3x_2x_3\\
x_2' &=  k_1x_1 - k_2x_2^2 - k_3x_2x_3\\
x_3' &=  k_2x_2^2
\end{align*}
Given \(k_1=0.04, k_2=3.0e7, k_3=1.0e4\), we make inference
regarding the initial condition for \(x_1(t=0)\).
\end{block}
\end{frame}
\begin{frame}[label={sec:orgcde0199}]{Nonlinear ODEs without analytical solution}
\begin{align*}
x_1' &= -k_1x_1 + k_3x_2x_3\\
x_2' &=  k_1x_1 - k_2x_2^2 - k_3x_2x_3\\
x_3' &=  k_2x_2^2
\end{align*}
Given \(k_1=0.04, k_2=3.0e7, k_3=1.0e4\), we make inference
regarding the initial condition for \(x_1(t=0)\).
\begin{block}{Exercise}
Write Stan function for the above ODE's RHS.
\end{block}
\end{frame}

\begin{frame}[fragile,label={sec:org92255f4}]{Stan function for autocatalytic kinetics}
 \begin{align*}
x_1' &= -k_1x_1 + k_3x_2x_3\\
x_2' &=  k_1x_1 - k_2x_2^2 - k_3x_2x_3\\
x_3' &=  k_2x_2^2
\end{align*}

\begin{minted}[breaklines=true,fontsize=\footnotesize,breakanywhere=true]{stan}
functions{
  real[] reaction(real t, real[] x, real[] theta, real[] r, int[] i){
    real dxdt[3];
    real k1 = theta[1];
    real k2 = theta[2];
    real k3 = theta[3];
    dxdt[1] = -k1*x[1] + k3*x[2]*x[3];
    dxdt[2] =  k1*x[1] - k3*x[2]*x[3] - k2*(x[2])^2;
    dxdt[3] =  k2*(x[2])^2;
    return dxdt;
  }
}
\end{minted}
\begin{itemize}
\item What's the initial conditions for \(x_2\) and \(x_3\)?
\end{itemize}
\end{frame}

\begin{frame}[fragile,label={sec:org9200f08}]{Numerical integrators}
 \begin{itemize}
\item Runge-Kutta 4th/5th (\texttt{rk45})
\begin{itemize}
\item non-stiff equations
\item Most popular, try this if you don't know the nature of the ODE, or what you're doing, or both.
\end{itemize}
\item Backward differentiation formula (\texttt{bdf})
\begin{itemize}
\item stiff equations
\item More expensive to use
\end{itemize}
\item Adams-Moulton
\begin{itemize}
\item non-stiff equations
\item higher-order of accuracy(do you really need it?)
\item scales better with number of steps
\end{itemize}
\end{itemize}
\end{frame}

\begin{frame}[fragile,label={sec:org19dc52e}]{Numerical integrators}
 \begin{center}
\begin{tabular}{lll}
Integrators & Stan & Torsten\\
\hline
\texttt{rk45} & \texttt{integrate\_ode\_rk45} & \texttt{pmx\_integrate\_ode\_rk45}\\
\texttt{BDF} & \texttt{integrate\_ode\_bdf} & \texttt{pmx\_integrate\_ode\_bdf}\\
\texttt{Adams} & \texttt{integrate\_ode\_adams} & \texttt{pmx\_integrate\_ode\_adams}\\
\end{tabular}

\end{center}

\begin{minted}[breaklines=true,fontsize=\footnotesize,breakanywhere=true]{stan}
real[ , ] pmx_integrate_ode_rk45(ODE_RHS, real[] y0, real t0, real[] ts, real[] theta, real[] x_r, int[] x_i, real rtol = 1.e-6, real atol = 1.e-6, int max_step = 1e6);
\end{minted}
\begin{itemize}
\item \texttt{ODE\_RHS}: ODE right-hand-side \(f\) in \(y' = f(y, t, \theta, x_r, x_i)\).
\item \texttt{y0}: initial condition at time \texttt{t0}.
\item \texttt{t0}: initial time.
\item \texttt{ts}: times at which we require a solution.
\item \texttt{theta}: parameters to be passed to \(f\).
\item \texttt{x\_r}: real data to be passed to \(f\).
\item \texttt{x\_i}: integer data to be passed to \(f\).
\end{itemize}
\end{frame}


\begin{frame}[fragile,label={sec:orgd71c300}]{Exercise}
 \begin{itemize}
\item In each of 8 experiments performed \(x3\) is observed.
\item Hierarchical model for \(x0[1]\)
\end{itemize}
\begin{minted}[breaklines=true,fontsize=\footnotesize,breakanywhere=true]{stan}
model {
  y0_mu ~ lognormal(log(2.0), 0.5);
  for (i in 1:nsub) {
    y0_1[i] ~ lognormal(y0_mu, 0.5);    
  }
  sigma ~ cauchy(0, 0.5); 
  obs ~ lognormal(log(x3), sigma);
}
\end{minted}
\end{frame}

\begin{frame}[fragile,label={sec:orga1635e8}]{Exercise}
 \begin{block}{Data available for the inference}
\begin{minted}[breaklines=true,fontsize=\footnotesize,breakanywhere=true]{stan}
data {
  int<lower=1> nsub;       /* nb. of subjects */
  int<lower=1> len[nsub];  /* nb. of results-extraction time points for each subject */
  int<lower=1> ntot;       /* total nb. of results-extraction time points */
  real ts[ntot];           /* concatenated array for results-extraction time points */
  real obs[ntot];          /* concatenated array for observed x3 */
}
\end{minted}
\end{block}
\end{frame}

\begin{frame}[fragile,label={sec:orgd04565b}]{Exercise}
 \begin{block}{Given above data and model, write the rest of Stan code.}
\begin{itemize}
\item Hint: use \texttt{chem.stan} as template, also see \texttt{chem.data.R} and \texttt{chem.init.R}.
\item Reaction begins with \(A\)(on which is also what we'd
like to make inference), the other two spiecies are
non-existent at the beginning of the reaction.
\item Which numerical integrator are you using? Why?
\end{itemize}
\end{block}
\end{frame}

\begin{frame}[fragile,label={sec:org664be57}]{Exercise}
 How to build \& run?
\begin{block}{Edit/Add \texttt{cmdstan/make/local}}
\begin{minted}[breaklines=true,fontsize=\footnotesize,breakanywhere=true]{sh}
TORSTEN_MPI = 1  # flag on torsten's MPI solvers
CXXFLAGS += -isystem /usr/local/include    # path to MPI library's headers
\end{minted}
\end{block}
\begin{block}{Build in \texttt{cmdstan}}
\begin{minted}[breaklines=true,fontsize=\footnotesize,breakanywhere=true]{sh}
make path_to_workshop/RScript/model/chemical_reactions/chem
\end{minted}
\end{block}
\begin{block}{Run}
\begin{minted}[breaklines=true,fontsize=\footnotesize,breakanywhere=true]{sh}
./chem sample adapt delta=0.95 random seed=1104508041 data file=chem.data.R init=chem.init.R
\end{minted}
\end{block}
\end{frame}

\section{Population models | \footnotesize{Charles Margossian}}
\label{sec:org348d308}
\label{orgfbc82e8}
\begin{frame}[label={sec:orga3653d0}]{Data pooled into groups}
\begin{itemize}
\item sport measurements are grouped by players
\item people's voting intention can be grouped by states, social status, etc.
\item medical measurements are grouped by patients
\end{itemize}
\end{frame}

\begin{frame}[fragile,label={sec:org5af9837}]{Data pooled into groups}
 \begin{itemize}
\item medical measurements are grouped by patients
\begin{itemize}
\item Simulated with \texttt{mrgsolve} \url{https://mrgsolve.github.io/}
\end{itemize}
\end{itemize}
\begin{figure}
  \includegraphics[width = 7.5cm]{Dosing_regimes.png}
\end{figure}
\end{frame}

\begin{frame}[label={sec:orgc4be33f}]{Hierarchical model}
With a hierarchical model, we can
\begin{itemize}
\item do partial pooling.
\item estimate how similar the groups are to one another.
\item estimate individual parameters.
$$\theta = (\theta_1, ..., \theta_L) \sim p(\theta | \theta_\mathrm{pop}) $$
$$y = (y_1, ..., y_N) \sim p(y | \theta, x) $$
\end{itemize}
\end{frame}
\begin{frame}[label={sec:org9244ff8}]{Hierarchical model}
$$\theta = (\theta_1, ..., \theta_L) \sim p(\theta | \theta_\mathrm{pop}) $$
$$y = (y_1, ..., y_N) \sim p(y | \theta, x) $$
\begin{center}
\includegraphics[width=0.6\textwidth]{./figures/hierachical_model.pdf}
\end{center}
\end{frame}

\begin{frame}[label={sec:orgadc9dd2}]{Example 3: Hierarchical two compartment model}
Likelihood function:
\begin{align*}
  &\log \theta \sim \mathrm{Normal}(\log \theta_\mathrm{pop}, \Omega) \\ \\
  &\Omega = \left(\begin{array}{ccccc} 
                          \omega_1 & 0 & 0 & 0 & 0 \\
                          0 & \omega_2 & 0 & 0 & 0 \\
                          0 & 0 & \omega_3 & 0 & 0 \\
                          0 & 0 & 0 & \omega_4 & 0 \\
                          0 & 0 & 0 & 0 & \omega_5
                          \end{array} \right) \\ \\ \\
  &\log (cObs) \sim \mathrm{Normal}\left(\log \left(\frac{y_2}{VC} \right), \sigma^2 \right)
\end{align*}
\end{frame}

\begin{frame}[label={sec:org5db0e99}]{Exercise 6: Write, fit, and diagnose a hierarchical two}
compartment model for a population of 10 patients.
Use \texttt{data/twoCptPop.data.r} and \texttt{twoCptPop.r}.\}
\begin{itemize}
\item \emph{Start by running 3 chains with 30 iterations.}
\item \emph{Do you get any warning messages?}
\end{itemize}
\end{frame}
\begin{frame}[fragile,label={sec:orga563c75}]{Divergent transitions}
 \begin{itemize}
\item \emph{Do you get any warning messages?}
\begin{minted}[breaklines=true,fontsize=\footnotesize,breakanywhere=true]{bash}
There were 29 divergent transitions after warmup.
\end{minted}
\item A divergent transition occurs when we fail to accurately compute a Hamiltonian trajectory.
\item This is because we \textit{approximate} trajectories.
\item Our sampler may not be refined enough to explore the entire typical set.
\end{itemize}
\end{frame}

\begin{frame}[label={sec:orgc6b5715}]{Divergent transitions}
Consider the following hierarchical model:

\begin{align*}
  \alpha_i \sim& \mathrm{Normal}(\mu, \sigma)  \\
  y_i \sim& p(y | \alpha_i)
\end{align*}
\end{frame}

\begin{frame}[label={sec:org6fdb299}]{Divergent transitons}
$$ \alpha_i \sim \mathrm{Normal}(\mu, \sigma) $$

Fitting this model yields the following pairs plot:
\begin{center}
  \includegraphics[width=5cm]{pairs_baseball2.png}
\end{center}
\end{frame}

\begin{frame}[label={sec:org6671460}]{Divergent transitons}
\begin{center}
  \includegraphics[width=5cm]{pairs_baseball2.png}
\end{center}

\begin{itemize}
\item This geometric shape is known as Neil's funnel \cite{Neil:2003}.
\item Its interactions with HMC is described in \cite{Betancourt:2015}.
\item It occurs in hierarchical models when we have sparse data and
a centered prior.
\end{itemize}
\end{frame}
\begin{frame}[label={sec:org6529b26}]{Reparameterization}
\begin{block}{Proposition}
Reparameterize the model to avoid the funnel shape.
We will do so by standardizing \(\alpha\).

$$ \alpha_{\mathrm{std}, i} := \frac{\alpha_i - \mu}{\sigma} $$

Then

$$ \alpha_\mathrm{std} \sim \mathrm{Normal}(0, 1)  $$
\end{block}
\end{frame}
\begin{frame}[label={sec:org70c8f6a}]{Reparameterization}
Then
  $$ \alpha_i = \mu + \sigma \alpha_{\mathrm{std}, i} $$

Hence
  $$ y_i  \sim p(\mu + \sigma \alpha_{\mathrm{std}, i}) $$
\begin{itemize}
\item Same data generating process; but how does this affect the geometry of the posterior?
\end{itemize}
\end{frame}
\begin{frame}[label={sec:org9f07c30}]{Reparameterization}
Our model is a little more complicated than the above example:
\begin{itemize}
\item a lot of parameters (100 +)!
\item multiple population parameters and hierarchical structures.
\item these parameters follow a log normal distribution (so we need a pairs plot with \(\log \theta\)).
\end{itemize}
\end{frame}

\begin{frame}[label={sec:org4a5b9ee}]{Reparameterization}
\begin{center}
  \includegraphics[width=9cm]{twoCptPairs1}
\end{center}
\end{frame}

\begin{frame}[label={sec:org37490bd}]{Reparameterization}
\emph{\textcolor{MRGGreen}{Exercise 6}}:
  Reparametrize the two compartment population model and fit it.\}
\begin{itemize}
\item First, work out the appropriate parametrization. You should start with \(\log \theta_i \sim \mathrm{Normal}(\theta_{\mathrm{pop}, i}, \omega)\)
\item Write, fit, and check the inference (run 100 chains).
\item What kind of predictive checks can we do?
\end{itemize}
\end{frame}

\begin{frame}[label={sec:org7ce9cc2}]{Reparameterization}
Need:
\begin{itemize}
\item predictions at an individual level
\item predictions at a population level

As always, this comes down to properly writing the data generating process
in the generated quantities block.
\end{itemize}
\end{frame}
\begin{frame}[label={sec:org3faec43}]{Individual predictions}
\begin{center}
  \includegraphics[width = 10cm]{PredictionPatient.png}
\end{center}
\end{frame}

\begin{frame}[label={sec:org13c304a}]{Population predictions}
\begin{center}
  \includegraphics[width = 10cm]{PredictionPopulation.png}
\end{center}
\end{frame}

\begin{frame}[label={sec:orgc0f3653}]{Further reading}
For a very good case study on hierarchical models, see,
Bob Carpenter's \emph{Pooling with Hierarchical Models for Repeated Binary Trials}

\url{https://mc-stan.org/users/documentation/case-studies/pool-binary-trials.html}
\end{frame}

\section{ODE group integrators | \footnotesize{Yi Zhang}}
\label{sec:orge9499a2}

\begin{frame}[fragile,label={sec:org80b8f49}]{ODE group integrators}
 \begin{center}
\begin{tabular}{ll}
Single ODE system & ODE group\\
\hline
\texttt{pmx\_integrate\_ode\_rk45} & \texttt{pmx\_integrate\_ode\_group\_rk45}\\
\texttt{pmx\_integrate\_ode\_bdf} & \texttt{pmx\_integrate\_ode\_group\_bdf}\\
\texttt{pmx\_integrate\_ode\_adams} & \texttt{pmx\_integrate\_ode\_group\_adams}\\
\end{tabular}

\end{center}

\begin{columns}
\begin{column}{0.45\columnwidth}
\begin{block}{Single ODE system}
\begin{minted}[breaklines=true,fontsize=\footnotesize,breakanywhere=true]{stan}
real[,]
pmx_integrate_ode_xxx(
      f,
      real[] y0, real t0,
      real[] ts,
      real[] theta,
      real[] x_r, int[] x_i,
      ...);
\end{minted}
\end{block}
\end{column}

\begin{column}{0.55\columnwidth}
\begin{block}{ODE group}
\begin{minted}[breaklines=true,fontsize=\footnotesize,breakanywhere=true]{stan}
matrix
pmx_integrate_ode_group_xxx(
     f,
     real[ , ] y0, real t0,
     int[] len, real[] ts,
     real[ , ] theta,
     real[ , ] x_r, int[ , ] x_i,
     ...);
\end{minted}
\end{block}
\end{column}
\end{columns}
\end{frame}

\begin{frame}[fragile,label={sec:org568c93d}]{ODE group integrators}
 \begin{columns}
\begin{column}{0.45\columnwidth}
\begin{block}{Single ODE system}
\begin{minted}[breaklines=true,fontsize=\footnotesize,breakanywhere=true]{stan}
real[ , ]
pmx_integrate_ode_xxx(
      f,
      real[] y0, real t0,
      real[] ts,
      real[] theta,
      real[] x_r, int[] x_i,
      ...);
\end{minted}
\end{block}
\end{column}

\begin{column}{0.55\columnwidth}
\begin{block}{ODE group}
\begin{minted}[breaklines=true,fontsize=\footnotesize,breakanywhere=true]{stan}
matrix
pmx_integrate_ode_group_xxx(
     f,
     real[ , ] y0, real t0,
     int[] len, real[] ts,
     real[ , ] theta,
     real[ , ] x_r, int[ , ] x_i,
     ...);
\end{minted}
\end{block}
\end{column}
\end{columns}
\begin{block}{}
\begin{itemize}
\item \texttt{len} specifies the length of data for each subject within
the above ragged arrays, and the size of \texttt{len} is the size
of the population.
\item The group integrators return a single matrix ragged
column-wise. The number of rows equals to the size of ODE system.
\end{itemize}
\end{block}
\end{frame}

\begin{frame}[label={sec:orge4d0f76}]{Exercise}
\begin{block}{autocatalytic reaction model: ODE group version}
\begin{itemize}
\item Change the loop with the numerical integrator to use group
integrator.
\item Remeber the return of the group integrator is a matrix
\begin{itemize}
\item nb. of rows: nb. of states
\item nb. of cols: nb. of \emph{total} results-extraction time points.
\end{itemize}
\end{itemize}
\end{block}
\end{frame}

\begin{frame}[fragile,label={sec:orgcdff97e}]{Exercise}
 \begin{block}{Build and run}
\begin{itemize}
\item Edit/Add \texttt{cmdstan/make/local}
\end{itemize}
\begin{minted}[breaklines=true,fontsize=\footnotesize,breakanywhere=true]{sh}
TORSTEN_MPI = 1  # flag on torsten's MPI solvers
CXXFLAGS += -isystem /usr/local/include    # path to MPI library's headers
\end{minted}
\begin{itemize}
\item Build in \texttt{cmdstan}
\end{itemize}
\begin{minted}[breaklines=true,fontsize=\footnotesize,breakanywhere=true]{sh}
make ../example-models/chemical_reactions/chem_group
\end{minted}
\begin{itemize}
\item Run
\end{itemize}
\begin{minted}[breaklines=true,fontsize=\footnotesize,breakanywhere=true]{sh}
mpiexec -n 2 -l ./chem_group sample adapt delta=0.95 random seed=1104508041 data file=chem.data.R init=chem.init.R
\end{minted}
\end{block}
\end{frame}

\begin{frame}[fragile,label={sec:orgba1e2dc}]{Exercise}
 \begin{itemize}
\item What does output say?
\item How many cores can you use until performance saturates? Why?
\item (optional)Can you do it using Stan's \texttt{map\_rect}? Is there a difference in style, output, and performance?
\end{itemize}
\end{frame}

\section{PMX population solvers | \footnotesize{Yi Zhang}}
\label{sec:org3da596c}
\label{org5b56f03}
\begin{frame}[fragile,label={sec:orgc18622b}]{PMX population solvers}
 \begin{center}
\begin{tabular}{ll}
Single ODE system & ODE group\\
\hline
\texttt{pmx\_solve\_rk45} & \texttt{pmx\_solve\_group\_rk45}\\
\texttt{pmx\_solve\_bdf} & \texttt{pmx\_solve\_group\_bdf}\\
\texttt{pmx\_solve\_adams} & \texttt{pmx\_solve\_group\_adams}\\
\end{tabular}

\end{center}

\begin{columns}
\begin{column}{0.45\columnwidth}
\begin{block}{Individual solvers}
\begin{minted}[breaklines=true,fontsize=\footnotesize,breakanywhere=true]{stan}
matrix
pmx_solve_bdf(f, int nCmt,
  real[] time, real[] amt,
  real[] rate, real[] ii,
  int[] evid, int[] cmt,
  real[] addl, int[] ss,
  real[] theta, real[] biovar,
  real[] tlag, real rel_tol,
  real abs_tol, int max_step);
\end{minted}
\end{block}
\end{column}

\begin{column}{0.55\columnwidth}
\begin{block}{Population solvers}
\begin{minted}[breaklines=true,fontsize=\footnotesize,breakanywhere=true]{stan}
matrix
pmx_solve_group_bdf(f, int nCmt,
  int[] len, real[] time,
  real[] amt, real[] rate,
  real[] ii, int[] evid,
  int[] cmt, real[] addl,
  int[] ss, real[ , ] theta,
  real[ , ] biovar, real[ , ] tlag,
  real rel_tol, real abs_tol,
  int max_step);
\end{minted}
\end{block}
\end{column}
\end{columns}
\end{frame}


\begin{frame}[fragile,label={sec:org6e33a20}]{PMX population solvers}
 \begin{minted}[breaklines=true,fontsize=\footnotesize,breakanywhere=true]{stan}
matrix
pmx_solve_group_bdf(f, int nCmt, int[] len, real[] time, real[] amt, real[] rate, real[] ii, int[] evid, int[] cmt, real[] addl, int[] ss, real[,] theta, real[,] biovar, real[,] tlag, real rel_tol, real abs_tol, int max_step);
\end{minted}

\begin{block}{}
\begin{figure}[htbp]
\centering
\includegraphics[width=0.6\textwidth]{./figures/group_solver_args.pdf}
\caption{arguments and output of \texttt{pmx\_solve\_group\_xxx}}
\end{figure}
\end{block}
\end{frame}

\begin{frame}[label={sec:org82db13b}]{Exercise}
We analyze the time to the first grade 2+ peripheral neuropathy
(PN) event in patients treated with an antibody-drug conjugate (ADC) delivering monomethyl auristatin E
(MMAE). We will simulate and analyze data using a simplified version of the
model reported in \cite{lu_time--event_2017}.
\begin{itemize}
\item Fauxlatuzumab vedotin 1.2 mg/kg IV boluses q3w \(\times\) 6 does.
\item 19 patients with 6 right-censored (simulated data).
\end{itemize}
\begin{columns}
\begin{column}{0.3\columnwidth}
\begin{block}{Model scheme}
\begin{center}
\includegraphics[width=0.9\columnwidth]{./figures/lu2017Model.pdf}
\end{center}
\end{block}
\end{column}
\begin{column}{0.7\columnwidth}
\begin{block}{Note}
\begin{itemize}
\item To keep things simpler, we use the simulated individual CL and V values, and only model PD part of the problem.
\item PN hazard is substantially delayed relative to PK exposure.
\item Hazard increases over time to an extent not completely described by P
\end{itemize}
\end{block}
\end{column}
\end{columns}
\end{frame}
\begin{frame}[label={sec:org2ea6c8a}]{Exercise}
Likelihood for time to first PN \(\ge\) 2 event in the \(i^{th}\) patient:
\begin{align*}
\lefteqn{L\left(\theta | t_{\text{PN},i}, \text{censor}_i, X_i\right)} \\
  &= \left\{ \begin{array}{ll}
     h_i\left(t_{\text{PN},i} | \theta, X_i\right) e^{-\int_0^{t_{\text{PN},i}} h_i\left(u | \theta, X_i\right) du}, &
    \text{censor}_i = 0 \\
     e^{-\int_0^{t_{\text{PN},i}} h_i\left(u | \theta, X_i\right) du}, &
     \text{censor}_i = 1
\end{array} \right.
\end{align*}
where
 \begin{align*}
   t_{\text{PN}} &\equiv \text{time to first PN $\ge$ 2 or right
     censoring event} \\
 \theta &\equiv \text{model parameters} \\
 X &\equiv \text{independent variables / covariates} \\
 \text{censor} &\equiv \left\{ \begin{array}{ll}
     1, & \text{PN $\ge$ 2 event is right censored} \\
     0, & \text{PN $\ge$ 2 event is observed} 
 \end{array} \right.
\end{align*}
One can see the expression
\begin{equation*}
  e^{-\int_0^{t_{\text{PN},i}} h_i\left(u | \theta, X_i\right) du}
\end{equation*}
as the survival function at time \(t\).
\end{frame}

\begin{frame}[label={sec:org9a0a57f}]{Exercise}
Hazard of PN grade 2+ based on the Weibull distribution,
with drug effect proportional to effect site concentration of MMAE:
\begin{align*}
  h_j(t) &= \beta E_{\text{drug}j}(t)^\beta t^{(\beta - 1)} \\
  E_{\text{drug}j}(t) &= \alpha c_{ej}(t) \\
  c^\prime_{ej}(t) &= k_{e0} \left(c_j(t) - c_{ej}(t)\right).
\end{align*}

Overall ODE system including integration of the hazard function:
\begin{align}
  x_1^\prime &= -\frac{CL}{V} x_1 \\
  x_2^\prime &= k_{e0} \left(\frac{x_1}{V} - x_2\right) \\
  x_3^\prime &= h(t)
  \end{align}
where \(x_2(t) = c_e(t)\) and \(x_3(t) = \int_0^t h(u) du\) aka cumulative hazard.
\end{frame}

\begin{frame}[fragile,label={sec:org10423f0}]{Exercise}
 \begin{block}{"just walk in a minute ago, literally" mode}
Apply \texttt{pmx\_solve\_group\_rk45} function
\end{block}
\begin{block}{Intermediate mode}
Code \texttt{pmx\_solve\_group\_rk45} function and its args. Use input data file \texttt{ttp2n.data2.R} as hint.
\end{block}
\begin{block}{hard mode}
Code ODE, \texttt{pmx\_solve\_group\_rk45} function and its args,
and the likelihood for harzard and censor event. Use input
data file \texttt{ttp2n.data2.R} and \texttt{model} block as hint.
\end{block}
\begin{block}{"why bother" mode}
\end{block}
\end{frame}

\begin{frame}[fragile,label={sec:orgb2b40f3}]{Exercise}
 \begin{block}{Edit/Add \texttt{cmdstan/make/local}}
\begin{minted}[breaklines=true,fontsize=\footnotesize,breakanywhere=true]{sh}
TORSTEN_MPI = 1         # flag on torsten's MPI solvers
CXXFLAGS += -isystem /usr/local/include # path to MPI library's headers
\end{minted}
\end{block}
\begin{block}{Build in \texttt{cmdstan}}
\begin{minted}[breaklines=true,fontsize=\footnotesize,breakanywhere=true]{sh}
make ../example-models/ttpn2/ttpn2_group
\end{minted}
\end{block}
\begin{block}{Run}
\begin{minted}[breaklines=true,fontsize=\footnotesize,breakanywhere=true]{sh}
mpiexec -n 4 -l ttpn2_group sample num_warmup=500 num_samples=500 data file=ttpn2.data2.R init=ttpn2.init.R
\end{minted}
\end{block}
\end{frame}

\begin{frame}[fragile,label={sec:org0a495e4}]{Exercise}
 \begin{itemize}
\item The parallel performance is not optimal, why?
\item Can you do it using Stan's \texttt{map\_rect}?
\end{itemize}
\end{frame}

\section{Additional nonlinear ODE example | \footnotesize{Charles Margossian}}
\label{sec:org90fbb75}
\begin{frame}[label={sec:orgfb4bbbb}]{Friberg-Karlsson semi-mechanistic model \cite{Friberg:2002}}
\begin{center}
  \includegraphics[width=8.5cm]{Friberg-Karlsson_drug}
\end{center}
\end{frame}

\begin{frame}[label={sec:org620ba4e}]{ODE system of F-K model}
\begin{align*}\label{eq:FK}
y_{\mathrm{prol}}' &= k_{\mathrm{tr}} y_{\mathrm{prol}} (1 - {\color{red}E_{\mathrm{drug}}})\left(\frac{Circ_0}
  {y_{\mathrm{circ}}}\right)^\gamma - k_{\mathrm{tr}}y_{\mathrm{prol}} \\
y_{\mathrm{tr1}}' &= k_{\mathrm{tr}} y_{\mathrm{prol}} - k_{\mathrm{tr}} y_{\mathrm{tr1}} \\
y_{\mathrm{tr2}}' &= k_{\mathrm{tr}} y_{\mathrm{tr1}} - k_{\mathrm{tr}} y_{\mathrm{tr2}} \\
y_{\mathrm{tr3}}' &= k_{\mathrm{tr}} y_{\mathrm{tr2}} - k_{\mathrm{tr}} y_{\mathrm{tr3}} \\
y_{\mathrm{circ}}' &= k_{\mathrm{tr}} y_{\mathrm{tr3}} - k_{\mathrm{tr}} y_{\mathrm{circ}} 
\end{align*}

where \(E_\mathrm{drug} = \alpha \frac{y_{\mathrm{cent}}}{V_{\mathrm{cent}}}\),
\(ktr = 4 / MTT\),
and \(\alpha \approx 3e-4\).
\begin{itemize}
\item \(y_\mathrm{cent}\) is obtained from a two compartment model.
\item Our PK/PD model therefore has a total of 8 equations.
\item This problem can be solved using \texttt{pmx\_solve\_*}.
\end{itemize}
\end{frame}
\begin{frame}[fragile,label={sec:org3609edb}]{Coupled PK-PD system}
 Alternatively, we may elect to solve the PK ODEs \textcolor{MRGGreen}{analytically} 
and the PD ODEs \textcolor{MRGGreen}{numerically}.
\begin{itemize}
\item This can yield some speedup, in particular for problems that require ODE solutions and sensitivities (e.g \cite{Margossian:2017b}).
\end{itemize}
\begin{minted}[breaklines=true,fontsize=\footnotesize,breakanywhere=true]{stan}
real[] pmx_solve_twocpt_rk45(reduced_ODE_system, int nOde, real[] time, real[] amt, real[] rate, real[] ii, int[] evid, int[] cmt, real[] addl, int[] ss, real[] theta, real[] biovar, real[] tlag, real rel_tol, real abs_tol, real max_step)
\end{minted}
\end{frame}

\begin{frame}[fragile,label={sec:org303198c}]{Coupled PK-PD system}
 \begin{minted}[breaklines=true,fontsize=\footnotesize,breakanywhere=true]{stan}
real[] pmx_solve_twocpt_rk45(reduced_ODE_system, int nOde, real[] time, real[] amt, real[] rate, real[] ii, int[] evid, int[] cmt, real[] addl, int[] ss, real[] theta, real[] biovar, real[] tlag, real rel_tol, real abs_tol, real max_step)
\end{minted}
\begin{itemize}
\item we now pass a "reduced system".
\item we specify the number of ODEs to be solved numerically, not the number of compartments.
\item \texttt{theta} now contains the parameters for the two cpt model, followed by the parameters that get passed to the numerical solver:
\end{itemize}
\begin{minted}[breaklines=true,fontsize=\footnotesize,breakanywhere=true]{stan}
theta = {CL, Q, VC, VP, ka, /* ... */ };
\end{minted}
\end{frame}

\begin{frame}[fragile,label={sec:org6ec30fa}]{Reduced system}
 \begin{minted}[breaklines=true,fontsize=\footnotesize,breakanywhere=true]{stan}
real[] reduced_system(real time, real[] y, real[] yPK, real[] theta, real[] x_r, int[] x_i) {
  real[3] dydt;
  /* .... */
  return dydt;
}
\end{minted}
\emph{\textcolor{MRGGreen}{Exercise 4 (optional)}: Write, fit, and diagnose a Friberg-Karlsson model with a two compartment with first order absorption PK. Use \texttt{FKModel.r} and \texttt{data/FKModel.data.r}.}
\end{frame}
\begin{frame}[label={sec:org3fdc078}]{Exercise}
\emph{Write, fit, and diagnose a Friberg-Karlsson model with a two compartment with first order absorption PK. Use \texttt{FKModel.r} and \texttt{data/FKModel.data.r}.}
\begin{itemize}
\item You may either use \texttt{pmx\_solve\_*} or \texttt{pmx\_solve\_twocpt\_*}.
\item Use \(\alpha = 3e-4\) and estimate all other 8 ODE coefficients,
i.e. \(\theta = \{ CL, Q, VC, VP, ka, MTT, circ0, \gamma \}\).
\item The initial state for the neutrophil count is \(Circ_0\). 
Either edit the event schedule to reflect this at time 0, 
or write the solution to your ODEs as a deviation from the baseline.
\end{itemize}
\end{frame}
\begin{frame}[label={sec:org92cdd61}]{Exercise}
\begin{itemize}
\item This exercise entails a few subtleties; in the interest of time we won't go through it in class.
\item Here are however results I get from 3 chains with 500 iterations you can use as a benchmark.
\end{itemize}
\end{frame}
\begin{frame}[label={sec:org7820b59}]{Exercise}
\begin{center}
  \includegraphics[width=7cm]{FKModelPlots002.pdf}
\end{center}
\end{frame}

\begin{frame}[label={sec:orgf7f6421}]{Exercise}
\begin{center}
  \includegraphics[width=7cm]{FKPairs}
\end{center}
\end{frame}

\begin{frame}[label={sec:org39d0bc5}]{Exercise}
\begin{center}
  \includegraphics[width=7cm]{FKModelPlots006.pdf}
\end{center}
\end{frame}

\begin{frame}[allowframebreaks]{Reference}
\bibliographystyle{apalike}
\footnotesize \bibliography{./ref}
\end{frame}
\end{document}
