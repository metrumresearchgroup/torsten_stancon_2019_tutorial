\documentclass[12pt]{article}

\usepackage{multicol}

%\usepackage{graphicx}
%\usepackage{float}
%\usepackage{apacite}


\begin{document}


\noindent \\
\textbf{
{\Large Population and ODE-based models \\ using Stan and Torsten}}

\ \\
\textit{Instructors:} Charles Margossian and Yi Zhang \\
\textit{Length}: 1 day  \\

\ \\
\textbf{Description:} \\
This class focuses on fitting models in pharmacometrics.
We consider the following challenges when building such models: 
(i) the data generating process involves solutions to ODE systems,
(ii) these ODEs are embedded in a complicated event schedule,
and (iii) the data comes from various sources, for instance various patients and studies,
and the resulting models are hierarchical.
Note that these properties are not specific to pharmacometrics,
and arise in other fields such as epidemiology, geology, and econometrics.
To accommodate a broad audience, we will keep the core concepts general,
and review basic notions of pharmacometrics, so that participants from all fields
can do the exercises.

The course covers elementary techniques to solve ODEs in Stan,
the efficient parametrization of hierarchical models,
and within-chain parallelization.
We also introduce \textit{Torsten}, an extension of Stan for pharmacometrics,
which allows us to seamlessly combine the above methods.

\ \\
\textbf{Prerequesites:} \\
Participants should be familiar with Bayesian statistics and basic Stan.
The requisite material is covered in the introductory tutorial and the tutorial on hierarchical modeling
at Stan Con 2019, which participants can attend before coming to this workshop.

\ \\
\textbf{Outline:} \\
%
  \begin{enumerate}
  \item Course information
   \item Introduction
   \begin{itemize}
     \item Modeling framework: build, fit, and criticize
     \item Review: diagnosing inference
     \item Review: criticizing the model
   \end{itemize}
   %
   \item Models in pharmacometrics
     \begin{itemize}
       \item Compartment models
       \item The event schedule
       \item Torsten: a library of Stan functions for pharmacometrics
       \item \textit{Exercise 1: build, fit, and diagnose a two compartment model}
     \end{itemize}
   %
   %
   \item Ordinary differential equations in Stan and Torsten
   \begin{itemize}
     \item Arsenal of tools to solve ODEs
     \item Matrix exponential solution for linear ODEs
     \item Numerical integrators for nonlinear ODEs
     \item \textit{Exercise 2: write, fit, and diagnose a two compartment 
     model with the ODE integrator or Matrix exponential solver}
   \end{itemize}
   %
   %
   \item Numerical ODE integrators
   \begin{itemize}
     \item example: kinetics of autocatalytic reaction
     \item \textit{Exercise 3: specify ODE system for autocatalytic reaction model}
     \item Numerical integrators: rk45, bdf, and Adams-Moulton
     \item \textit{Exercise 4: build and fit the full autocatalytic reaction model}
   \end{itemize}
   %
   %
   \item Population models
   \begin{itemize}
     \item Review of hierarchical models
     \item \textit{Exercise 5: write, fit, and diagnose a population two compartment model}
     \item Divergent transitions and where they come from
     \item \textit{Exercise 6: re-parametrize the population two compartment model}
   \end{itemize}
   %
   \item ODE group integrators
   \begin{itemize}
     \item group integrators in Torsten
     \item \textit{Exercise 6: parallelize the autocatalytic reaction model using the group integrator}
   \end{itemize}
   %
   \item PMX population solvers
   \begin{itemize}
     \item Time to event model
     \item \textit{Exercise 7: specify the ODE system for time to event model.}
     \item \textit{Exercise 8: use the group solver to parallel solve the ODE}
   \end{itemize}
   \item Open discussion and concluding remarks.
 \end{enumerate}


\end{document}
