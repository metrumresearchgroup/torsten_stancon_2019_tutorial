% Created 2019-07-30 Tue 13:36
% Intended LaTeX compiler: pdflatex
\documentclass[presentation]{beamer}
\usepackage[utf8]{inputenc}
\usepackage[T1]{fontenc}
\usepackage{graphicx}
\usepackage{grffile}
\usepackage{longtable}
\usepackage{wrapfig}
\usepackage{rotating}
\usepackage[normalem]{ulem}
\usepackage{amsmath}
\usepackage{textcomp}
\usepackage{amssymb}
\usepackage{capt-of}
\usepackage{hyperref}
\usepackage[newfloat]{minted}
\usepackage{caption}
\usetheme{default}
\author{Yi Zhang, Charles Margossian}
\date{\today}
\title{Numerical ODE integrators in Stan/Torsten}
\hypersetup{
 pdfauthor={Yi Zhang, Charles Margossian},
 pdftitle={Numerical ODE integrators in Stan/Torsten},
 pdfkeywords={},
 pdfsubject={},
 pdfcreator={Emacs 25.3.1 (Org mode 9.1.3)}, 
 pdflang={English}}
\begin{document}

\maketitle

\begin{frame}[fragile,label={sec:orgf38b6c7}]{Nonlinear ODEs without analytical solution}
 \begin{minted}[breaklines=true,fontsize=\footnotesize,breakanywhere=true]{stan}
functions {
  real[] reaction(real t, real[] x, real[] p, real[] r, int[] i){
    /* k1r = k_-1 */
    real k1  = p[1];
    real k1r = p[2];
    real k2  = p[3];
    real S  = x[1];
    real E  = x[2];
    real ES = x[3];
    real P  = x[4];
    real dSdt  = k1r * ES - k1 * S * E;
    real dEdt  = k1r * ES + k2 * ES - k1 * S * E;
    real dESdt = k1 * S * E - k1r * ES - k2 * ES;
    real dPdt  = k2 * ES;

    return {dSdt, dEdt, dESdt, dPdt};
  }
}
\end{minted}
\end{frame}

\begin{frame}[fragile,label={sec:orgd154d4a}]{Numerical integrators}
 \begin{itemize}
\item Runge-Kutta 4th/5th (\texttt{rk45})
\begin{itemize}
\item non-stiff equations
\item Most popular, try this if you don't know the nature of the ODE, or what you're doing, or both.
\end{itemize}
\item Backward differentiation formula (\texttt{bdf})
\begin{itemize}
\item stiff equations
\item More expensive to use
\end{itemize}
\item Adams-Moulton
\begin{itemize}
\item non-stiff equations
\item higher-order of accuracy(do you really need it?)
\item scales better with number of steps
\end{itemize}
\end{itemize}
\end{frame}

\begin{frame}[fragile,label={sec:orgef46c12}]{Numerical integrators}
 \begin{center}
\begin{tabular}{lll}
Integrators & Stan & Torsten\\
\hline
\texttt{rk45} & \texttt{integrate\_ode\_rk45} & \texttt{pmx\_integrate\_ode\_rk45}\\
\texttt{BDF} & \texttt{integrate\_ode\_bdf} & \texttt{pmx\_integrate\_ode\_bdf}\\
\texttt{Adams} & \texttt{integrate\_ode\_adams} & \texttt{pmx\_integrate\_ode\_adams}\\
\end{tabular}

\end{center}

\begin{minted}[breaklines=true,fontsize=\footnotesize,breakanywhere=true]{stan}
real[ , ] pmx_integrate_ode_rk45(ODE_RHS, real[] y0, real t0, real[] ts, real[] theta, real[] x_r, int[] x_i, real rtol = 1.e-6, real atol = 1.e-6, int max_step = 1e6);
\end{minted}
\begin{itemize}
\item \texttt{ODE\_RHS}: ODE right-hand-side \(f\) in \(y' = f(y, t, \theta, x_r, x_i)\).
\item \texttt{y0}: initial condition at time \texttt{t0}.
\item \texttt{t0}: initial time.
\item \texttt{ts}: times at which we require a solution.
\item \texttt{theta}: parameters to be passed to \(f\).
\item \texttt{x\_r}: real data to be passed to \(f\).
\item \texttt{x\_i}: integer data to be passed to \(f\).
\end{itemize}
\end{frame}

\begin{frame}[label={sec:orgbd2556c}]{Exercise}
We consider the kinetics of an autocatalytic reaction \cite{robertson_numerical_1966}. The
structure of the reactions is 
\begin{align*}
A &\xrightarrow{k_1} B\\
B+B &\xrightarrow{k_2} C + B\\
B+C&\xrightarrow{k_3} C + A,
\end{align*}
where \(k_1\), \(k_2\), \(k_3\) are the rate
constants and \(A\), \(B\) and \(C\) are the chemical species
involved. The corresponding ODEs are
\begin{align*}
x_1' &= -k_1x_1 + k_3x_2x_3\\
x_2' &=  k_1x_1 - k_2y_2^2 - k_3x_2x_3\\
x_3' &=  k_2y_2^2
\end{align*}
Given \(k_1=0.04, k_2=1.0e4, k_3=3.0e7\), we seek the
initial condition for \(x_1(t=0)\).
\end{frame}

\begin{frame}[fragile,label={sec:org39fd9ad}]{Exercise}
 \begin{itemize}
\item Write the \texttt{functions} block for \(f\).
\item What's the initial conditions for \(x_2\) and \(x_3\)?
\item Which numerical integrator to use? Why? Did you try other options?
\end{itemize}
\end{frame}

\begin{frame}[fragile,label={sec:org3946b70}]{Exercise}
 How to build \& run?
\begin{block}{Edit/Add \texttt{cmdstan/make/local}}
\begin{minted}[breaklines=true,fontsize=\footnotesize,breakanywhere=true]{sh}
TORSTEN_MPI = 1  # flag on torsten's MPI solvers
CXXFLAGS += -isystem /usr/local/include    # path to MPI library's headers
\end{minted}
\end{block}
\begin{block}{Build in \texttt{cmdstan}}
\begin{minted}[breaklines=true,fontsize=\footnotesize,breakanywhere=true]{sh}
make ../example-models/ttpn2/ttpn2_group
\end{minted}
\end{block}
\begin{block}{Run}
\begin{minted}[breaklines=true,fontsize=\footnotesize,breakanywhere=true]{sh}
./chem sample adapt delta=0.95 random seed=1104508041 data file=chem.data.R init=chem.init.R
\end{minted}
\end{block}
\end{frame}

\begin{frame}[label={sec:org33a04d3}]{Reference}
\bibliography{../../examples/chemical_reactions/autocatalysis}
\bibliographystyle{plain}
\end{frame}
\end{document}
