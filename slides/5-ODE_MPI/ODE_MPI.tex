% Created 2019-07-30 Tue 14:05
% Intended LaTeX compiler: pdflatex
\documentclass[presentation]{beamer}
\usepackage[utf8]{inputenc}
\usepackage[T1]{fontenc}
\usepackage{graphicx}
\usepackage{grffile}
\usepackage{longtable}
\usepackage{wrapfig}
\usepackage{rotating}
\usepackage[normalem]{ulem}
\usepackage{amsmath}
\usepackage{textcomp}
\usepackage{amssymb}
\usepackage{capt-of}
\usepackage{hyperref}
\usepackage[newfloat]{minted}
\usepackage{caption}
\usetheme{default}
\author{Yi Zhang}
\date{\today}
\title{Numerical ODE group/population integrators in Torsten}
\hypersetup{
 pdfauthor={Yi Zhang},
 pdftitle={Numerical ODE group/population integrators in Torsten},
 pdfkeywords={},
 pdfsubject={},
 pdfcreator={Emacs 25.3.1 (Org mode 9.1.3)}, 
 pdflang={English}}
\begin{document}

\maketitle

\begin{frame}[fragile,label={sec:org498f494}]{ODE group integrators}
 \begin{center}
\begin{tabular}{ll}
Single ODE system & ODE group\\
\hline
\texttt{pmx\_integrate\_ode\_rk45} & \texttt{pmx\_integrate\_ode\_group\_rk45}\\
\texttt{pmx\_integrate\_ode\_bdf} & \texttt{pmx\_integrate\_ode\_group\_bdf}\\
\texttt{pmx\_integrate\_ode\_adams} & \texttt{pmx\_integrate\_ode\_group\_adams}\\
\end{tabular}

\end{center}

\begin{columns}
\begin{column}{0.45\columnwidth}
\begin{block}{Single ODE system}
\begin{minted}[breaklines=true,fontsize=\footnotesize,breakanywhere=true]{stan}
real[,]
pmx_integrate_ode_xxx(
      f,
      real[] y0, real t0,
      real[] ts,
      real[] theta,
      real[] x_r, int[] x_i,
      ...);
\end{minted}
\end{block}
\end{column}

\begin{column}{0.55\columnwidth}
\begin{block}{ODE group}
\begin{minted}[breaklines=true,fontsize=\footnotesize,breakanywhere=true]{stan}
matrix
pmx_integrate_ode_group_xxx(
     f,
     real[ , ] y0, real t0,
     int[] len, real[] ts,
     real[ , ] theta,
     real[ , ] x_r, int[ , ] x_i,
     ...);
\end{minted}
\end{block}
\end{column}
\end{columns}
\end{frame}

\begin{frame}[fragile,label={sec:org887e05c}]{ODE group integrators}
 \begin{columns}
\begin{column}{0.45\columnwidth}
\begin{block}{Single ODE system}
\begin{minted}[breaklines=true,fontsize=\footnotesize,breakanywhere=true]{stan}
real[ , ]
pmx_integrate_ode_xxx(
      f,
      real[] y0, real t0,
      real[] ts,
      real[] theta,
      real[] x_r, int[] x_i,
      ...);
\end{minted}
\end{block}
\end{column}

\begin{column}{0.55\columnwidth}
\begin{block}{ODE group}
\begin{minted}[breaklines=true,fontsize=\footnotesize,breakanywhere=true]{stan}
matrix
pmx_integrate_ode_group_xxx(
     f,
     real[ , ] y0, real t0,
     int[] len, real[] ts,
     real[ , ] theta,
     real[ , ] x_r, int[ , ] x_i,
     ...);
\end{minted}
\end{block}
\end{column}
\end{columns}
\begin{block}{}
\begin{itemize}
\item \texttt{len} specifies the length of data for each subject within
the above ragged arrays, and the size of \texttt{len} is the size
of the population.
\item The group integrators return a single matrix ragged
column-wise. The number of rows equals to the size of ODE system.
\end{itemize}
\end{block}
\end{frame}

\begin{frame}[fragile,label={sec:org7f849b8}]{Exercise}
 \begin{block}{autocatalytic reaction model: ODE group version}
\begin{itemize}
\item Change the loop with the numerical integrator to use group
integrator.

\item Edit/Add \texttt{cmdstan/make/local}
\end{itemize}
\begin{minted}[breaklines=true,fontsize=\footnotesize,breakanywhere=true]{sh}
TORSTEN_MPI = 1  # flag on torsten's MPI solvers
CXXFLAGS += -isystem /usr/local/include    # path to MPI library's headers
\end{minted}
\begin{itemize}
\item Build in \texttt{cmdstan}
\end{itemize}
\begin{minted}[breaklines=true,fontsize=\footnotesize,breakanywhere=true]{sh}
make ../example-models/ttpn2/ttpn2_group
\end{minted}
\begin{itemize}
\item Run
\end{itemize}
\begin{minted}[breaklines=true,fontsize=\footnotesize,breakanywhere=true]{sh}
mpiexec -n 2 -l ./chem_group sample adapt delta=0.95 random seed=1104508041 data file=chem.data.R init=chem.init.R
\end{minted}
\end{block}
\end{frame}

\begin{frame}[fragile,label={sec:org10c48aa}]{Exercise}
 \begin{itemize}
\item What does output say?
\item How many cores can you use until performance saturates? Why?
\item Can you do it using Stan's \texttt{map\_rect}? Is there a
difference in style, output, and performance?
\end{itemize}
\end{frame}
\end{document}
